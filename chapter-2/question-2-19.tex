        % ------------------------------------------ %
        %                      (2.19)                %
        % ------------------------------------------ %
        Let $\underset{\left(m \times m\right)}{\bold{A}^{1/2}} = \sum_{i=1}^m{\sqrt{\lambda_i} \bold{e}_i \bold{e}_i^\prime} = \bold{P}\bold{\Lambda}^{1/2}\bold{P}^\prime$, where $\bold{P}\bold{P}^\prime = \bold{P}^\prime\bold{P} = \bold{I}$. (The $\lambda_i$'s and the $\bold{e}$'s are the eigenvalues and associated normalized eigenvectors of the matrix $\bold{A}$.) Show Properties \textbf{(1)-(4)} of the square-root matrix in \textbf{(2-22)}.
        \begin{enumerate}
            \item[\textbf{(1)}]{${\left(\bold{A}^{1/2}\right)}^\prime = \bold{A}^{1/2}$ (that is, $\bold{A}$ is symmetric).}
            
            \[
                {\left(\bold{A}^{1/2}\right)}^\prime
                =
                {\left(\bold{P}\bold{\Lambda}^{1/2}\bold{P}^\prime\right)}^\prime
                \overset{\text{Exercise 2.3(c)}}{=}
                {\left(\bold{P}^\prime\right)}^\prime {\left(\bold{\Lambda}^{1/2}\right)}^\prime {\left(\bold{P}\right)}^\prime
                =
            \]
            \[
                =
                \bold{P} \bold{\Lambda}^{1/2} \bold{P}^\prime
                =
                \bold{A}^{1/2}
            \]

            \item[\textbf{(2)}]{$\bold{A}^{1/2}\bold{A}^{1/2} = \bold{A}$.}
            
            \[
                \bold{A}^{1/2}\bold{A}^{1/2} = 
                \left(\bold{P}\bold{\Lambda}^{1/2}\bold{P}^\prime\right)\left(\bold{P}\bold{\Lambda}^{1/2}\bold{P}^\prime\right)
                =
                \bold{P}\bold{\Lambda}^{1/2}\left(\bold{P}^\prime\bold{P}\right)\bold{\Lambda}^{1/2}\bold{P}^\prime
                =
            \]
            \[
                =
                \bold{P}\bold{\Lambda}^{1/2}\bold{I}\bold{\Lambda}^{1/2}\bold{P}^\prime
                =
                \bold{P}\bold{\Lambda}^{1/2}\bold{\Lambda}^{1/2}\bold{P}^\prime
                =
            \]
            \[
                =
                \bold{P}
                \begin{bmatrix}
                    \sqrt{\lambda_1} & 0 & \cdots & 0 \\
                    0 & \sqrt{\lambda_2} & \cdots & 0 \\
                    \vdots & \vdots & \ddots & \vdots \\
                    0 & 0 & \cdots & \sqrt{\lambda_m}
                \end{bmatrix}
                \begin{bmatrix}
                    \sqrt{\lambda_1} & 0 & \cdots & 0 \\
                    0 & \sqrt{\lambda_2} & \cdots & 0 \\
                    \vdots & \vdots & \ddots & \vdots \\
                    0 & 0 & \cdots & \sqrt{\lambda_m}
                \end{bmatrix}
                \bold{P}^\prime
                =
            \]
            \[
                \bold{P}
                \begin{bmatrix}
                    \lambda_1 & 0 & \cdots & 0 \\
                    0 & \lambda_2 & \cdots & 0 \\
                    \vdots & \vdots & \ddots & \vdots \\
                    0 & 0 & \cdots & \lambda_m
                \end{bmatrix}
                \bold{P}^\prime
                =
                \bold{P}
                \bold{\Lambda}
                \bold{P}^\prime
                =
                \bold{A}
            \]

            \item[\textbf{(3)}]{${\left(\bold{A}^{1/2}\right)}^{-1} = \sum_{i=1}^k{\frac{1}{\sqrt{\lambda_i}}\bold{e}_i\bold{e}_i^\prime} = \bold{P}\bold{\Lambda}^{-1/2}\bold{P}^\prime$, where $\bold{\Lambda}^{-1/2}$ is a diagonal matrix with $1/\sqrt{\lambda_i}$ as the ith diagonal element.}
            \newline
            \par
            First off, something useful,            
            \[
                \bold{\Lambda}^{-1/2}\bold{\Lambda}^{1/2}
                =
                \begin{bmatrix}
                    1/\sqrt{\lambda_1} & 0 & \cdots & 0 \\
                    0 & 1/\sqrt{\lambda_2} & \cdots & 0 \\
                    \vdots & \vdots & \ddots & \vdots \\
                    0 & 0 & \cdots & 1/\sqrt{\lambda_m}
                \end{bmatrix}
                \begin{bmatrix}
                    \sqrt{\lambda_1} & 0 & \cdots & 0 \\
                    0 & \sqrt{\lambda_2} & \cdots & 0 \\
                    \vdots & \vdots & \ddots & \vdots \\
                    0 & 0 & \cdots & \sqrt{\lambda_m}
                \end{bmatrix}
                =
                \bold{I}
            \]

            \[
                \bold{\Lambda}^{1/2}\bold{\Lambda}^{-1/2}
                =
                \begin{bmatrix}
                    \sqrt{\lambda_1} & 0 & \cdots & 0 \\
                    0 & \sqrt{\lambda_2} & \cdots & 0 \\
                    \vdots & \vdots & \ddots & \vdots \\
                    0 & 0 & \cdots & \sqrt{\lambda_m}
                \end{bmatrix}
                \begin{bmatrix}
                    1/\sqrt{\lambda_1} & 0 & \cdots & 0 \\
                    0 & 1/\sqrt{\lambda_2} & \cdots & 0 \\
                    \vdots & \vdots & \ddots & \vdots \\
                    0 & 0 & \cdots & 1/\sqrt{\lambda_m}
                \end{bmatrix}
                =
                \bold{I}
            \]

            We have $\bold{\Lambda}^{-1/2}\bold{\Lambda}^{1/2} = \bold{\Lambda}^{1/2}\bold{\Lambda}^{-1/2} = \bold{I}$, so $\bold{\Lambda}^{-1/2} = {\left(\bold{\Lambda}^{1/2}\right)}^{-1}$ is the inverse of $\bold{\Lambda}^{1/2}$ by \textbf{Definition 2A.27}.

            \[
                {\left(\bold{A}^{1/2}\right)}^{-1}
                =
                {\left(\bold{P}\bold{\Lambda}^{1/2}\bold{P}^\prime\right)}^{-1}
                \overset{Exercise 2.4(b)}{=}
                {\left(\bold{P}^\prime\right)}^{-1}{\left(\bold{\Lambda}^{1/2}\right)}^{-1}\bold{P}^{-1} 
                =
            \]
            \[
                \overset{\bold{P}^\prime = \bold{P}^{-1}}{=}
                {\left(\bold{P}^\prime\right)}^\prime{\left(\bold{\Lambda}^{1/2}\right)}^{-1}\bold{P}^\prime
                =
                \bold{P}{\left(\bold{\Lambda}^{1/2}\right)}^{-1}\bold{P}^\prime
            \]

            \[
                =
                \bold{P}\bold{\Lambda}^{-1/2}\bold{P}^\prime
                =
                \sum_{i=1}^m{\frac{1}{\sqrt{\lambda_i}}\bold{e}_i\bold{e}_i^\prime}
            \]

            \item[\textbf{(4)}]{$\bold{A}^{1/2}\bold{A}^{-1/2} = \bold{A}^{-1/2}\bold{A}^{1/2} = \bold{I}$, and $\bold{A}^{-1/2}\bold{A}^{-1/2} = \bold{A}^{-1}$, where $\bold{A}^{-1/2} = {\left(\bold{A}^{1/2}\right)}^{-1}$.}
            \par
            Above in \textbf{(3)} it was shown that $\bold{A}^{1/2}\bold{A}^{-1/2} = \bold{A}^{-1/2}\bold{A}^{1/2} = \bold{I}$, so $\bold{A}^{-1/2} = {\left(\bold{A}^{1/2}\right)}^{-1}$.
            \[
                \bold{A}^{-1/2}\bold{A}^{-1/2} 
                = 
                \begin{bmatrix}
                    1/\sqrt{\lambda_1} & 0 & \cdots & 0 \\
                    0 & 1/\sqrt{\lambda_2} & \cdots & 0 \\
                    \vdots & \vdots & \ddots & \vdots \\
                    0 & 0 & \cdots & 1/\sqrt{\lambda_m}
                \end{bmatrix}
                \begin{bmatrix}
                    1/\sqrt{\lambda_1} & 0 & \cdots & 0 \\
                    0 & 1/\sqrt{\lambda_2} & \cdots & 0 \\
                    \vdots & \vdots & \ddots & \vdots \\
                    0 & 0 & \cdots & 1/\sqrt{\lambda_m}
                \end{bmatrix}
                =
            \]
            \[
                =
                \begin{bmatrix}
                    1/\lambda_1 & 0 & \cdots & 0 \\
                    0 & 1/\lambda_2 & \cdots & 0 \\
                    \vdots & \vdots & \ddots & \vdots \\
                    0 & 0 & \cdots & 1/\lambda_m
                \end{bmatrix}
                =
                \bold{A}^{-1}
            \]

        \end{enumerate}