        % ------------------------------------------ %
        %                      (2.9)                 %
        % ------------------------------------------ %
        Let $\bold{A}$ be as in Exercise 2.8.
        \begin{enumerate}[label=(\alph*)]
            \item Find $\bold{A}^{-1}$.
            \par
            Using \textbf{(2-21)}, 
            \[
                \bold{A}^{-1}
                =
                \bold{P}\bold{\Lambda}^{-1}\bold{P}^\prime
                =
                \begin{bmatrix}
                    1/\sqrt{5} & 2/\sqrt{5} \\
                    -2/\sqrt{5} & 1/\sqrt{5}
                \end{bmatrix}
                \begin{bmatrix}
                    -1/3 & 0 \\
                    0 & 1/2
                \end{bmatrix}
                \begin{bmatrix}
                    1/\sqrt{5} & -2/\sqrt{5} \\
                    2/\sqrt{5} & 1/\sqrt{5}
                \end{bmatrix}
                =
            \]
            \[
                =
                \frac{1}{5}
                \begin{bmatrix}
                    1 & 2 \\
                    -2 & 1
                \end{bmatrix}
                \begin{bmatrix}
                    -1/3 & 0 \\
                    0 & 1/2
                \end{bmatrix}
                \begin{bmatrix}
                    1 & -2 \\
                    2 & 1
                \end{bmatrix}
                =
                \frac{1}{5}
                \begin{bmatrix}
                    1 & 2 \\
                    -2 & 1
                \end{bmatrix}
                \begin{bmatrix}
                    -1/3 & 2/3 \\
                    1 & 1/2
                \end{bmatrix}
                =
            \]
            \[
                =
                \frac{1}{5}
                \begin{bmatrix}
                    5/3 & 5/3 \\
                    5/3 & -5/6
                \end{bmatrix}
                =
                \begin{bmatrix}
                    1/3 & 1/3 \\
                    1/3 & -1/6
                \end{bmatrix}
            \]
            Using direct computation,
            \[
                \bold{A}^{-1}
                =
                \frac{1}{-2-4}
                \begin{bmatrix}
                    -2 & -2 \\
                    -2 & 1
                \end{bmatrix}
                =
                \begin{bmatrix}
                    1/3 & 1/3 \\
                    1/3 & -1/6
                \end{bmatrix}
            \]
            \item Compute the eigenvalues and eigenvectors of $\bold{A}^{-1}$.
            \par
            Also from \textbf{(2-21)} on page 66, we can see that the eigenvalues of $\bold{A}^{-1}$ are the reciprocal of the eigenvalues of $\bold{A}$, so
            \[
                \bold{\Lambda}^{-1}
                =
                \begin{bmatrix}
                    1/\lambda_1 & 0 \\
                    0 & 1/\lambda_2
                \end{bmatrix}
                =
                \begin{bmatrix}
                    -1/3 & 0 \\
                    0 & 1/2
                \end{bmatrix}
            \]
            and the eigenvectors for $\bold{A}^{-1}$ are the same as those for $\bold{A}$,
            \[
                \bold{P}
                =
                \begin{bmatrix}
                    \bold{e}_1 & \bold{e}_2
                \end{bmatrix}
                =
                \begin{bmatrix}
                    1/\sqrt{5} & 2/\sqrt{5} \\
                    -2/\sqrt{5} & 1/\sqrt{5}
                \end{bmatrix}
            \]
            \item Write the spectral decomposition of $\bold{A}^{-1}$, and compare it with that of $\bold{A}$ from Exercise 2.8.
            \par
            \[
                \bold{A}^{-1} = \sum_{k=1}^2{\frac{1}{\lambda_k}\bold{e}_k\bold{e}_k^\prime} = 
                -\frac{1}{3}
                \begin{bmatrix}
                    1/\sqrt{5} \\
                    -2/\sqrt{5}
                \end{bmatrix}
                \begin{bmatrix}
                    1/\sqrt{5} & -2/\sqrt{5}
                \end{bmatrix}
                +
                \frac{1}{2}
                \begin{bmatrix}
                    2/\sqrt{5} \\
                    1/\sqrt{5}
                \end{bmatrix}
                \begin{bmatrix}
                    2/\sqrt{5} & 1/\sqrt{5}
                \end{bmatrix}
                =
            \]
            \[
                =
                \frac{1}{5}
                \left(
                -\frac{1}{3}
                \begin{bmatrix}
                    1 & -2 \\
                    -2 & 4
                \end{bmatrix}
                +
                \frac{1}{2}
                \begin{bmatrix}
                    4 & 2 \\
                    2 & 1
                \end{bmatrix}
                \right)
                =
                \frac{1}{30}
                \left(
                -
                \begin{bmatrix}
                    2 & -4 \\
                    -4 & 8
                \end{bmatrix}
                +
                \begin{bmatrix}
                    12 & 6 \\
                    6 & 3
                \end{bmatrix}
                \right)
                =
            \]
            \[
                =
                \frac{1}{30}
                \begin{bmatrix}
                    10 & 10 \\
                    10 & -5
                \end{bmatrix}
                =
                \begin{bmatrix}
                    1/3 & 1/3 \\
                    1/3 & -1/6
                \end{bmatrix}
            \]
        \end{enumerate}
        In the spectral decomposition of both $\bold{A}$ and $\bold{A}^{-1}$ the matrices 
        created for all of the $\bold{e}_k\bold{e}_k^\prime$ components are the same. 
        The difference is in the eigenvalues. The eigenvalues for $\bold{A}$ are $\lambda_k$ 
        and the eigenvalues for $\bold{A}^{-1}$ are $1/\lambda_k$. 