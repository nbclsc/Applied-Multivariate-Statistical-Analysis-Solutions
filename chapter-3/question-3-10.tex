When the generalized variance is zero, it is the columns of the mean corrected data matrix $\textbf{X}_c = \textbf{X} - \textbf{1}\bar{\textbf{x}}^\prime$ that are linearly dependent, 
not necessarily those of the data matrix itself Given the data
\[
    \textbf{X}
    =
    \begin{bNiceArray}{ccc}
        3 & 1 & 0 \\
        6 & 4 & 6 \\
        4 & 2 & 2 \\
        7 & 0 & 3 \\
        5 & 3 & 4
    \end{bNiceArray}
\]
\begin{enumerate}[label=(\alph*)]
    \item Obtain the mean corrected data matrix, and verify that the columns are linearly dependent.
    Specify an $\textbf{a}^{\prime} = [a_1, a_2, a_3]$ vector that establishes the linear dependence.
    \[
        \bar{\textbf{x}}
        =
        \begin{bNiceArray}{c}
            \bar{x}_1 \\
            \bar{x}_2 \\
            \bar{x}_3
        \end{bNiceArray}
        =
        \begin{bNiceArray}{c}
            25/5 \\
            10/5 \\
            15/5
        \end{bNiceArray}
        =
        \begin{bNiceArray}{c}
            5 \\
            2 \\
            3
        \end{bNiceArray}
    \]
    \[
        \textbf{X} - \textbf{1}_{5}{\bar{\textbf{x}}}^{\prime}
        =
        \begin{bNiceArray}{ccc}
            3 & 1 & 0 \\
            6 & 4 & 6 \\
            4 & 2 & 2 \\
            7 & 0 & 3 \\
            5 & 3 & 4
        \end{bNiceArray}
        -
        \begin{bNiceArray}{c}
            1 \\
            1 \\
            1 \\
            1 \\
            1
        \end{bNiceArray}
        \begin{bNiceArray}{ccc}
            5 & 2 & 3
        \end{bNiceArray}
        =
    \]
    \[
        =
        \begin{bNiceArray}{ccc}
            3 & 1 & 0 \\
            6 & 4 & 6 \\
            4 & 2 & 2 \\
            7 & 0 & 3 \\
            5 & 3 & 4
        \end{bNiceArray}
        -
        \begin{bNiceArray}{ccc}
            5 & 2 & 3 \\
            5 & 2 & 3 \\
            5 & 2 & 3 \\
            5 & 2 & 3 \\
            5 & 2 & 3
        \end{bNiceArray}
        =
        \begin{bNiceArray}{rrr}
            -2 & -1 & -3 \\
            1 & 2 & 3 \\
            -1 & 0 & -1 \\
            2 & -2 & 0 \\
            0 & 1 & 1
        \end{bNiceArray}
    \]
    For $\textbf{X}_{c}$ we have a linear dependence for column 3, whose the sum of the first two columns.
    \[
        \textbf{a}
        =
        \begin{bNiceArray}{c}
            a_1 \\
            a_2 \\
            a_3
        \end{bNiceArray}
        =
        \begin{bNiceArray}{r}
            1 \\
            1 \\
            -1
        \end{bNiceArray}
    \]
In Exercise 3.9. both $\textbf{X}_{c}\text{a} = \textbf{0}$ and $\textbf{X}\textbf{a} = \textbf{0}$ for the same $\textbf{a}$, but that isn't the case here. Here, $r(\textbf{X}) = p = 3$ so the columns of $\textbf{X}$ are linearly independent, and the columns of $\textbf{X}_c$ are linearly dependent ($r(\textbf{X}_c) = 2 < p = 3$).
    \[
        \left(\textbf{X} - \textbf{1}_{5}{\bar{\textbf{x}}}^{\prime}\right){\textbf{a}}
        =
        \begin{bNiceArray}{rrr}
            -2 & -1 & -3 \\
            1 & 2 & 3 \\
            -1 & 0 & -1 \\
            2 & -2 & 0 \\
            0 & 1 & 1
        \end{bNiceArray}
        \begin{bNiceArray}{r}
            1 \\
            1 \\
            -1
        \end{bNiceArray}
        =
        \begin{bNiceArray}{r}
            0 \\
            0 \\
            0 \\
            0 \\
            0
        \end{bNiceArray}
        =
        \textbf{0}
    \]
    \item Obtain the sample covariance matrix $\textbf{S}$, and verify that the generalized variance is zero.
    \[
        \textbf{D}
        =
        (\textbf{X} - \textbf{1}_{5}\bar{\textbf{x}})
    \]
    \[
        \textbf{S}
        =
        \left(\frac{1}{n - 1}\right){\textbf{D}}^{\prime}\textbf{D}
        =
        \left(\frac{1}{4}\right)
        \begin{bNiceArray}{rrrrr}
            -2 & 1 & -1 & 2 & 0 \\
            -1 & 2 & 0 & -2 & 1 \\
            -3 & 3 & -1 & 0 & 1 \\
        \end{bNiceArray}
        \begin{bNiceArray}{rrr}
            -2 & -1 & -3 \\
            1 & 2 & 3 \\
            -1 & 0 & -1 \\
            2 & -2 & 0 \\
            0 & 1 & 1
        \end{bNiceArray}
        =
    \]
    \[
        =
        \left(\frac{1}{4}\right)
        \begin{bNiceArray}{ccc}
            10 & 0 & 10 \\
            0 & 10 & 10 \\
            10 & 10 & 20
        \end{bNiceArray}
        =
        \begin{bNiceArray}{ccc}
            (5/2) & 0 & (5/2) \\
            0 & (5/2) & (5/2) \\
            (5/2) & (5/2) & 5
        \end{bNiceArray}
    \]
    Another way, using (3-27) on page 139,
    \[
        \textbf{S}
        =
        \left(\frac{1}{n - 1}\right)
        {\textbf{X}}^{\prime}\left(\textbf{I} - \frac{1}{n}\textbf{1}_{5}{\textbf{1}}_{5}^{\prime}\right)\textbf{X}
        =
        \left(\frac{1}{n - 1}\right)
        {\textbf{X}}^{\prime}\left(\textbf{X} - \frac{1}{n}\textbf{1}_{5}{\textbf{1}}_{5}^{\prime}\textbf{X}\right)
        =
    \]
    \[
        =
        \left(\frac{1}{n - 1}\right)
        {\textbf{X}}^{\prime}
        \left(\textbf{X} - \textbf{1}_{5}{\bar{\textbf{x}}}^{\prime}\right)
        =
        \left(\frac{1}{4}\right)
        \begin{bNiceArray}{rrrrr}
           3 & 6 & 4 & 7 & 5 \\
           1 & 4 & 2 & 0 & 3 \\
           0 & 6 & 2 & 3 & 4 \\
        \end{bNiceArray}
        \begin{bNiceArray}{rrr}
            -2 & -1 & -3 \\
            1 & 2 & 3 \\
            -1 & 0 & -1 \\
            2 & -2 & 0 \\
            0 & 1 & 1
        \end{bNiceArray}
        =
    \]
    \[
        =
        \left(\frac{1}{4}\right)
        \begin{bNiceArray}{ccc}
            10 & 0 & 10 \\
            0 & 10 & 10 \\
            10 & 10 & 20
        \end{bNiceArray}
        =
        \begin{bNiceArray}{ccc}
            (5/2) & 0 & (5/2) \\
            0 & (5/2) & (5/2) \\
            (5/2) & (5/2) & 5
        \end{bNiceArray}
    \]
    Computing the generalized sample variance
    \[
        \left|\textbf{S}\right|
        =
        \left|
            \begin{NiceArray}{ccc}
            (5/2) & 0 & (5/2) \\
            0 & (5/2) & (5/2) \\
            (5/2) & (5/2) & 5
            \end{NiceArray}
        \right|
        =
    \]
    \[
        =
        \left(\frac{5}{2}\right)
        \left|
            \begin{NiceArray}{cc}
                (5/2) & (5/2) \\
                (5/2) & 5
            \end{NiceArray}
        \right|
        -
        0
        +
        \left(\frac{5}{2}\right)
        \left|
            \begin{NiceArray}{ccc}
                0 & (5/2) \\
                (5/2) & (5/2)
            \end{NiceArray}
        \right|
        =
    \]
    \[
        =
        \frac{5}{8}
        (50-25)
        +
        \frac{5}{8}
        (0-25)
        =
    \]
    \[
        =
        \frac{125}{8}
        -
        \frac{125}{8}
        =
        0
    \]
    \item Show that the columns of the data matrix are linearly independent in this case.
    \par
    To show this, work $\textbf{X}$ into reduced row echelon form and count the pivot columns.
    \[
        \textbf{X}
        =
        \begin{bNiceArray}{ccc}
            3 & 1 & 0 \\
            6 & 4 & 6 \\
            4 & 2 & 2 \\
            7 & 0 & 3 \\
            5 & 3 & 4
        \end{bNiceArray}
        \overset{\text{Simplify rows}}{\longrightarrow}
        \begin{bNiceArray}{ccc}
            3 & 1 & 0 \\
            3 & 2 & 3 \\
            2 & 1 & 1 \\
            7 & 0 & 3 \\
            5 & 3 & 4
        \end{bNiceArray}
        \overset{\text{Row 5} - \frac{5}{3} \text{Row 1}}{\longrightarrow}
    \]
    \[
        \begin{bNiceArray}{ccc}
            3 & 1 & 0 \\
            3 & 2 & 3 \\
            2 & 1 & 1 \\
            7 & 0 & 3 \\
            0 & (4/3) & 4
        \end{bNiceArray}
        \overset{\text{Row 4} - \frac{7}{3} \text{Row 1}}{\longrightarrow}
        \begin{bNiceArray}{rrr}
            3 & 1 & 0 \\
            3 & 2 & 3 \\
            2 & 1 & 1 \\
            0 & -(7/3) & 3 \\
            0 & (4/3) & 4
        \end{bNiceArray}
        \overset{\text{Row 3} - \frac{2}{3} \text{Row 1}}{\longrightarrow}
    \]
    \[
        \begin{bNiceArray}{rrr}
            3 & 1 & 0 \\
            3 & 2 & 3 \\
            0 & (1/3) & 1 \\
            0 & -(7/3) & 3 \\
            0 & (4/3) & 4
        \end{bNiceArray}
        \overset{\text{Row 3} - \text{Row 1}}{\longrightarrow}
        \begin{bNiceArray}{rrr}
            3 & 1 & 0 \\
            0 & 1 & 3 \\
            0 & (1/3) & 1 \\
            0 & -(7/3) & 3 \\
            0 & (4/3) & 4
        \end{bNiceArray}
        \overset{\text{Simplify rows}}{\longrightarrow}
    \]
    \[
        \begin{bNiceArray}{rrr}
            3 & 1 & 0 \\
            0 & 1 & 3 \\
            0 & 1 & 3 \\
            0 & -7 & 9 \\
            0 & 1 & 3
        \end{bNiceArray}
        \overset{\text{Row 5} - \text{Row 2}}{\longrightarrow}
        \begin{bNiceArray}{rrr}
            3 & 1 & 0 \\
            0 & 1 & 3 \\
            0 & 1 & 3 \\
            0 & -7 & 9 \\
            0 & 0 & 0
        \end{bNiceArray}
        \overset{\text{Row 4} + 7 \text{Row 2}}{\longrightarrow}
    \]
    \[
        \begin{bNiceArray}{rrr}
            3 & 1 & 0 \\
            0 & 1 & 3 \\
            0 & 1 & 3 \\
            0 & 0 & 30 \\
            0 & 0 & 0
        \end{bNiceArray}
        \overset{\text{Row 3} - \text{Row 2}}{\longrightarrow}
        \begin{bNiceArray}{rrr}
            3 & 1 & 0 \\
            0 & 1 & 3 \\
            0 & 0 & 0 \\
            0 & 0 & 30 \\
            0 & 0 & 0
        \end{bNiceArray}
        \overset{\text{Swap rows}}{\longrightarrow}
    \]
    \[
        \begin{bNiceArray}{rrr}
            3 & 1 & 0 \\
            0 & 1 & 3 \\
            0 & 0 & 30 \\
            0 & 0 & 0 \\
            0 & 0 & 0
        \end{bNiceArray}
        \overset{\text{Simplify}}{\longrightarrow}
        \begin{bNiceArray}{rrr}
            3 & 1 & 0 \\
            0 & 1 & 3 \\
            0 & 0 & 1 \\
            0 & 0 & 0 \\
            0 & 0 & 0
        \end{bNiceArray}
        \overset{\text{Row 2} - 3\text{Row 3}}{\longrightarrow}
    \]
    \[
        \begin{bNiceArray}{rrr}
            3 & 1 & 0 \\
            0 & 1 & 0 \\
            0 & 0 & 1 \\
            0 & 0 & 0 \\
            0 & 0 & 0
        \end{bNiceArray}
        \overset{\text{Row 1} - \text{Row 2}}{\longrightarrow}
        \begin{bNiceArray}{rrr}
            3 & 0 & 0 \\
            0 & 1 & 0 \\
            0 & 0 & 1 \\
            0 & 0 & 0 \\
            0 & 0 & 0
        \end{bNiceArray}
        \overset{\text{Simplify}}{\longrightarrow}
    \]
    \[
        \begin{bNiceArray}{rrr}
            1 & 0 & 0 \\
            0 & 1 & 0 \\
            0 & 0 & 1 \\
            0 & 0 & 0 \\
            0 & 0 & 0
        \end{bNiceArray}
    \]
    We can see that in reduced row echelon form that there are 3 pivot columns (nonzero rows),
    so the rank of $\textbf{X}$ is 3 and is of full rank ($r(\textbf{X}) = 3 = p$).
\end{enumerate}