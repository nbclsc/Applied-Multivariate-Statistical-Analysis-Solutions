The following data matrix contains data on test scores, with $x_1 = \text{score on first test}$,
$x_2 = \text{score on second test}$, and $x_3 = \text{total score on the two tests}$:
\[
    \textbf{X}
    =
    \begin{bNiceArray}{ccc}
        12 & 17 & 29 \\
        18 & 20 & 38 \\
        14 & 16 & 30 \\
        20 & 18 & 38 \\
        16 & 19 & 35
    \end{bNiceArray}
\]
\begin{enumerate}[label=(\alph*)]
    \item Obtain the mean corrected data matrix, and verify that the columns are linearly dependent.
    Specify an $\textbf{a}^{\prime} = [a_1, a_2, a_3]$ vector that establishes the linear dependence.
    \[
        \bar{\textbf{x}}
        =
        \begin{bNiceArray}{c}
            \bar{x}_1 \\
            \bar{x}_2 \\
            \bar{x}_3
        \end{bNiceArray}
        =
        \begin{bNiceArray}{c}
            80/5 \\
            90/5 \\
            170/5
        \end{bNiceArray}
        =
        \begin{bNiceArray}{c}
            16 \\
            18 \\
            34
        \end{bNiceArray}
    \]
    \[
        \textbf{X} - \textbf{1}_{5}{\bar{\textbf{x}}}^{\prime}
        =
        \begin{bNiceArray}{ccc}
            12 & 17 & 29 \\
            18 & 20 & 38 \\
            14 & 16 & 30 \\
            20 & 18 & 38 \\
            16 & 19 & 35
        \end{bNiceArray}
        -
        \begin{bNiceArray}{c}
            1 \\
            1 \\
            1 \\
            1 \\
            1
        \end{bNiceArray}
        \begin{bNiceArray}{ccc}
            16 & 18 & 34
        \end{bNiceArray}
        =
    \]
    \[
        =
        \begin{bNiceArray}{ccc}
            12 & 17 & 29 \\
            18 & 20 & 38 \\
            14 & 16 & 30 \\
            20 & 18 & 38 \\
            16 & 19 & 35
        \end{bNiceArray}
        -
        \begin{bNiceArray}{ccc}
            16 & 18 & 34 \\
            16 & 18 & 34 \\
            16 & 18 & 34 \\
            16 & 18 & 34 \\
            16 & 18 & 34
        \end{bNiceArray}
        =
        \begin{bNiceArray}{rrr}
            -4 & -1 & -5 \\
            2 & 2 & 4 \\
            -2 & -2 & -4 \\
            4 & 0 & 4 \\
            0 & 1 & 1
        \end{bNiceArray}
    \]
    We have a linear dependence for column 3, whose the sum of the first two columns.
    \[
        \textbf{a}
        =
        \begin{bNiceArray}{c}
            a_1 \\
            a_2 \\
            a_3
        \end{bNiceArray}
        =
        \begin{bNiceArray}{r}
            1 \\
            1 \\
            -1
        \end{bNiceArray}
    \]
    \[
        \textbf{X}{\textbf{a}}
        =
        \begin{bNiceArray}{ccc}
            12 & 17 & 29 \\
            18 & 20 & 38 \\
            14 & 16 & 30 \\
            20 & 18 & 38 \\
            16 & 19 & 35
        \end{bNiceArray}
        \begin{bNiceArray}{r}
            1 \\
            1 \\
            -1
        \end{bNiceArray}
        =
        \begin{bNiceArray}{r}
            0 \\
            0 \\
            0 \\
            0 \\
            0
        \end{bNiceArray}
        =
        \textbf{0}
    \]
    or
    \[
        \left(\textbf{X} - \textbf{1}_{5}{\bar{\textbf{x}}}^{\prime}\right){\textbf{a}}
        =
        \begin{bNiceArray}{rrr}
            -4 & -1 & -5 \\
            2 & 2 & 4 \\
            -2 & -2 & -4 \\
            4 & 0 & 4 \\
            0 & 1 & 1
        \end{bNiceArray}
        \begin{bNiceArray}{r}
            1 \\
            1 \\
            -1
        \end{bNiceArray}
        =
        \begin{bNiceArray}{r}
            0 \\
            0 \\
            0 \\
            0 \\
            0
        \end{bNiceArray}
        =
        \textbf{0}
    \]
    \item Obtain the sample covariance matrix $\textbf{S}$, and verify that the generalized variance is zero.
    Also, show that $\textbf{S}\textbf{a} = \textbf{0}$, so a can be rescaled to be an eigenvector corresponding to eigenvalue zero.
    \[
        \textbf{D}
        =
        (\textbf{X} - \textbf{1}_{5}\bar{\textbf{x}})
    \]
    \[
        \textbf{S}
        =
        \left(\frac{1}{n - 1}\right){\textbf{D}}^{\prime}\textbf{D}
        =
        \left(\frac{1}{4}\right)
        \begin{bNiceArray}{rrrrr}
            -4 & 2 & -2 & 4 & 0 \\
            -1 & 2 & -2 & 0 & 1 \\
            -5 & 4 & -4 & 4 & 1
        \end{bNiceArray}
        \begin{bNiceArray}{rrr}
            -4 & -1 & -5 \\
            2 & 2 & 4 \\
            -2 & -2 & -4 \\
            4 & 0 & 4 \\
            0 & 1 & 1
        \end{bNiceArray}
        =
    \]
    \[
        =
        \left(\frac{1}{4}\right)
        \begin{bNiceArray}{ccc}
            40 & 12 & 52 \\
            12 & 10 & 22 \\
            52 & 22 & 74
        \end{bNiceArray}
        =
        \begin{bNiceArray}{ccc}
            10 & 3 & 13 \\
            3 & (5/2) & (11/2) \\
            13 & (11/2) & (37/2)
        \end{bNiceArray}
    \]
    Another way, using (3-27) on page 139,
    \[
        \textbf{S}
        =
        \left(\frac{1}{n - 1}\right)
        {\textbf{X}}^{\prime}\left(\textbf{I} - \frac{1}{n}\textbf{1}_{5}{\textbf{1}}_{5}^{\prime}\right)\textbf{X}
        =
        \left(\frac{1}{n - 1}\right)
        {\textbf{X}}^{\prime}\left(\textbf{X} - \frac{1}{n}\textbf{1}_{5}{\textbf{1}}_{5}^{\prime}\textbf{X}\right)
        =
    \]
    \[
        =
        \left(\frac{1}{n - 1}\right)
        {\textbf{X}}^{\prime}
        \left(\textbf{X} - \textbf{1}_{5}{\bar{\textbf{x}}}^{\prime}\right)
        =
        \left(\frac{1}{4}\right)
        \begin{bNiceArray}{rrrrr}
            12 & 18 & 14 & 20 & 16 \\
            17 & 20 & 16 & 18 & 19 \\
            29 & 38 & 30 & 38 & 35
        \end{bNiceArray}
        \begin{bNiceArray}{rrr}
            -4 & -1 & -5 \\
            2 & 2 & 4 \\
            -2 & -2 & -4 \\
            4 & 0 & 4 \\
            0 & 1 & 1
        \end{bNiceArray}
        =
    \]
    \[
        =
        \left(\frac{1}{4}\right)
        \begin{bNiceArray}{ccc}
            40 & 12 & 52 \\
            12 & 10 & 22 \\
            52 & 22 & 74
        \end{bNiceArray}
        =
        \begin{bNiceArray}{ccc}
            10 & 3 & 13 \\
            3 & (5/2) & (11/2) \\
            13 & (11/2) & (37/2)
        \end{bNiceArray}
    \]
    Computing the generalized sample variance
    \[
        \left|\textbf{S}\right|
        =
        \left|
            \begin{NiceArray}{ccc}
                10 & 3 & 13 \\
                3 & (5/2) & (11/2) \\
                13 & (11/2) & (37/2)
            \end{NiceArray}
        \right|
        =
    \]
    \[
        =
        10
        \left|
            \begin{NiceArray}{cc}
                (5/2) & (11/2) \\
                (11/2) & (37/2)
            \end{NiceArray}
        \right|
        -
        3
        \left|
            \begin{NiceArray}{cc}
                3 & (11/2) \\
                13 & (37/2)
            \end{NiceArray}
        \right|
        +
        13
        \left|
            \begin{NiceArray}{ccc}
                3 & (5/2) \\
                13 & (11/2)
            \end{NiceArray}
        \right|
        =
    \]
    \[
        =
        \frac{10}{4}
        (185-121)
        -
        \frac{3}{2}
        (111-143)
        +
        \frac{13}{2}
        (33-65)
        =
    \]
    \[
        =
        \frac{640}{4}
        +
        \frac{96}{2}
        -
        \frac{416}{2}
        =
    \]
    \[
        =
        \frac{416}{2}
        -
        \frac{416}{2}
        =
        0
    \]
    In part (a) we could see that the third column is the sum of the first two, so we defined a vector $\textbf{a}$ as
    \[
        \textbf{a}
        =
        \begin{bNiceArray}{r}
            1 \\
            1 \\
            -1
        \end{bNiceArray}
    \]
    and using that same vector to compute $\textbf{S}\textbf{a}$,
    \[
        \textbf{S}\textbf{a}
        =
        \begin{bNiceArray}{ccc}
            10 & 3 & 13 \\
            3 & (5/2) & (11/2) \\
            13 & (11/2) & (37/2)
        \end{bNiceArray}
        \begin{bNiceArray}{r}
            1 \\
            1 \\
            -1
        \end{bNiceArray}
        =
        \begin{bNiceArray}{c}
            13 - 13 \\
            11/2 - 11/2 \\
            37/2 - 37/2
        \end{bNiceArray}
        =
        \textbf{0}
    \]
    \item Verify that the third column of the data matrix is the sum of the first two columns.
    That is, show that there is linear dependence, with $a_1 = 1$, $a_2 = 1$, and $a_3 = -1$.
    \par
    In part (a) this was shown for the computation of $\textbf{X}\textbf{a} = \textbf{0}$. Here's another way using column vectors in $\textbf{X}$
    \[
        \textbf{X}\textbf{a}
        =
        \begin{bNiceArray}{ccc}
            \textbf{x}_1 & \textbf{x}_2 & \textbf{x}_3
        \end{bNiceArray}
        \begin{bNiceArray}{r}
            a_1 \\
            a_2 \\
            a_3
        \end{bNiceArray}
        =
        a_1 \textbf{x}_1
        +
        a_2 \textbf{x}_2
        +
        a_3 \textbf{x}_3
        =
        \textbf{x}_1
        +
        \textbf{x}_2
        -
        \textbf{x}_3
        =
    \]
    \[
        =
        \begin{bNiceArray}{c}
            12 \\
            18 \\
            14 \\
            20 \\
            16 \\
        \end{bNiceArray}
        +
        \begin{bNiceArray}{c}
            17 \\
            20 \\
            16 \\
            18 \\
            19 \\
        \end{bNiceArray}
        -
        \begin{bNiceArray}{c}
            29 \\
            38 \\
            30 \\
            38 \\
            35 \\
        \end{bNiceArray}
        =
        \begin{bNiceArray}{c}
            29 \\
            38 \\
            30 \\
            38 \\
            35 \\
        \end{bNiceArray}
        -
        \begin{bNiceArray}{c}
            29 \\
            38 \\
            30 \\
            38 \\
            35 \\
        \end{bNiceArray}
        =
        \begin{bNiceArray}{c}
            0 \\
            0 \\
            0 \\
            0 \\
            0 \\
        \end{bNiceArray}
        =
        \textbf{0}
    \]
\end{enumerate}