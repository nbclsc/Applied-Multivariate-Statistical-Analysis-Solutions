The data in Table 4.6 (see the psychological profile data: www.prenhall.com/statistics) 
consist of 130 observations generated by scores on a psychological test administered to Peruvian
teenagers (ages 15, 16, and 17). For each of these teenagers the gender (male = 1,
female = 2) and socioeconomic status (low = 1, medium = 2) were also recorded The
scores were accumulated into five subscale scores labeled \textit{independence} (indep), \textit{support}
(supp), \textit{benevolence} (benev), \textit{conformity} (conform), and \textit{leadership} (leader).

\begin{table}[H]
    \centering
    \begin{NiceTabular}[columns-width = 1.2cm]{|ccccccc|}
        \toprule
        \Block[l]{1-4}{\textbf{Table 4.6} Psychological Profile Data} & & \\
        Indep & Supp & Benev & Conform & Leader & Gender & Socio \\
        \midrule
        27 & 13 & 14 & 20 & 11 & 2 & 1 \\
        12 & 13 & 24 & 25 &  6 & 2 & 1 \\
        14 & 20 & 15 & 16 &  7 & 2 & 1 \\
        18 & 20 & 17 & 12 &  6 & 2 & 1 \\
         9 & 22 & 22 & 21 &  6 & 2 & 1 \\
        \vdots & \vdots & \vdots & \vdots & \vdots & \vdots & \vdots \\
        10 & 11 & 26 & 17 & 10 & 1 & 1 \\
        14 & 12 & 14 & 11 & 29 & 1 & 2 \\
        19 & 11 & 23 & 18 & 13 & 2 & 2 \\
        27 & 19 & 22 &  7 &  9 & 2 & 2 \\
        10 & 17 & 22 & 22 &  8 & 2 & 2 \\
        \Block{1-3}{\footnotesize Source: Data courtesy of C. Soto.} & & & & \\
        \bottomrule
        \CodeAfter~\tikz~\draw[solid] (2-|1) -- (2-|last);
        \tikz~\draw[solid] (14-|1) -- (14-|last);
    \end{NiceTabular}
\end{table}

\begin{enumerate}[label= (\alph*)]
    \item Examine each of the variables independence, support, benevolence, conformity and leadership for marginal normality.
    
    
    For ($x_{1}$), we're looking at the independence score for 130 valid observations.
    The simulated 0.01, 0.05, and 0.10 level critical correlation coefficient test values for a sample size of 76 are, 0.9860, 0.9900, and 0.9916, respectively.
    The Q-Q correlation coefficient using the raw data is 0.9881, which is larger than the 0.01-level critical point, but not the 0.05 and 0.10, and so would be considered normal at the 0.01-level but not the 0.05 and 0.10. 
    The Q-Q plot for the raw data is below. There is some slight curvature, but a transformation might help.

    \begin{center}
        \begin{figure}[H]
            \centering
            \includegraphics[scale=0.6]{./matlab/chapter-4/sol4.39.qq.1.png}
        \end{figure}
    \end{center}

    For ($x_{2}$), we're looking at the support score for 130 valid observations.
    The simulated 0.01, 0.05, and 0.10 level critical correlation coefficient test values for a sample size of 76 are, 0.9860, 0.9900, and 0.9916, respectively.
    The Q-Q correlation coefficient using the raw data is 0.9893, which is larger than the 0.01-level critical point, but not the 0.05 and 0.10, and so would be considered normal at the 0.01-level but not the 0.05 and 0.10. 
    The Q-Q plot for the raw data is below. There is some slight curvature, but a transformation also might help.

    \begin{center}
        \begin{figure}[H]
            \centering
            \includegraphics[scale=0.6]{./matlab/chapter-4/sol4.39.qq.2.png}
        \end{figure}
    \end{center}

    For ($x_{3}$), we're looking at the benevolence score for 130 valid observations.
    The simulated 0.01, 0.05, and 0.10 level critical correlation coefficient test values for a sample size of 76 are, 0.9860, 0.9900, and 0.9916, respectively.
    The Q-Q correlation coefficient using the raw data is 0.9925, which is larger than all three critical points, so the benevolence score would be considered normal at the 0.01, 0.05, and 0.10 levels. 
    The Q-Q plot for the raw data is below.
    There are five observations with the same score value of 29 causing a right tail effect, but overall $x_{3}$ looks marginally normal.

    \begin{center}
        \begin{figure}[H]
            \centering
            \includegraphics[scale=0.6]{./matlab/chapter-4/sol4.39.qq.3.png}
        \end{figure}
    \end{center}

    For ($x_{4}$), we're looking at the conformity score for 130 valid observations.
    The simulated 0.01, 0.05, and 0.10 level critical correlation coefficient test values for a sample size of 76 are, 0.9860, 0.9900, and 0.9916, respectively.
    The Q-Q correlation coefficient using the raw data is 0.9934, which is larger than all three critical points, so the conformity score would be considered normal at the 0.01, 0.05, and 0.10 levels. 
    The Q-Q plot for the raw data is below.
    There are three observations with the same score value of 27 causing a right tail effect, similar to $x_{3}$, but overall $x_{4}$ looks marginally normal.

    \begin{center}
        \begin{figure}[H]
            \centering
            \includegraphics[scale=0.6]{./matlab/chapter-4/sol4.39.qq.4.png}
        \end{figure}
    \end{center}

    For ($x_{5}$), we're looking at the leadership score for 130 valid observations.
    The simulated 0.01, 0.05, and 0.10 level critical correlation coefficient test values for a sample size of 76 are, 0.9860, 0.9900, and 0.9916, respectively.
    The Q-Q correlation coefficient using the raw data is 0.9813, which is smaller than all three critical values, and so would not be considered normal at the 0.01, 0.05, or 0.10 levels. 
    The Q-Q plot for the raw data is below. There is curvature in the plot. A transformation would absolutely help.

    \begin{center}
        \begin{figure}[H]
            \centering
            \includegraphics[scale=0.6]{./matlab/chapter-4/sol4.39.qq.5.png}
        \end{figure}
    \end{center}

    \item Using all five variables, check for multivariate normality.
    \item Refer to part (a). For those variables that are nonnormal, determine the transformation that makes them more nearly normal.
\end{enumerate}