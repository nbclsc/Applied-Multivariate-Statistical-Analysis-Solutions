(Example of a nonnormal bivariate distribution with normal marginals.) Let $X_1$ be $N(0,1)$ and let
\[
  X_2
  =
  \begin{cases*}
    -X_1 & if $-1 \leq X_1 \leq 1$ \\
    \phantom{-}X_1 & otherwise
  \end{cases*}
\]
Show each of the following.
\begin{enumerate}[label= (\alph*)]
    \item $X_2$ also has an $N(0,1)$ distribution.
    \par
    The hint is basically the answer. Using symmetry we can start with
    \[
      P (-1 < X_1 \leq x)
      =
      P(-x \leq X_1 < 1)
    \]
    Using the definition of the CDF for $X_2$
    \begin{align*}
      F_{X_2}(x_2)
      =
      \\
      =
      P (X_2 \leq x_2)
      =
      \\
      =
      P (X_2 \leq -1)
      +
      P (-1 < X_2 \leq x_2)
      = 
      \\
      =
      P (X_1 \leq -1)
      +
      P (-1 < -X_1 \leq x_2)
      = \tag*{Using the definition of $X_2$.}
      \\
      =
      P (X_1 \leq -1)
      +
      P (-x_2 \leq X_1 < 1)
      =
      \\
      =
      P (X_1 \leq -1)
      +
      P (-1 < -X_1 \leq x_2)
      = \tag*{Using symmetry argument.}
      \\
      P (X_1 \leq x_2) \tag*{CDF definition for $X_1$.}
      =
      \\
      F_{X_1}(x_2)
    \end{align*}
    The CDF for $X_1$ and $X_2$ are the same, so since $X_1 \sim N(0,1)$, then $X_2$ is also distributed as $N(0,1)$.
    
    \item $X_1$ and $X_2$ do \textit{not} have a bivariate normal distribution.
    \[
      \textbf{a}
      =
      \begin{bNiceArray}{r}
        1 \\
        -1
      \end{bNiceArray}
      ,\hspace{0.2in}
      \textbf{X}
      =
      \begin{bNiceArray}{r}
        X_1 \\
        X_2
      \end{bNiceArray}
      ,\hspace{0.2in}
      Y
      =
      \textbf{a}^{\prime}
      \textbf{X}
      =
      X_1 - X_2
    \]
    \[
        \textbf{a}^{\prime}
        \textbf{X}
      =
        X_1 - X_2
      =
      \begin{cases*}
        X_1 - (-X_1) & if $-1 \leq X_1 \leq 1$ \\
        X_1 - X_1 & otherwise
      \end{cases*}
      =
    \]
    \[
      =
      \begin{cases*}
        2X_1 & if $-1 \leq X_1 \leq 1$ \\
        0 & otherwise
      \end{cases*}
    \]
    If $X_1$ is in the interval $-1 \leq X_1 \leq 1$ then it's distribution as $ Y = (X_1 - X_2) \sim N(0,4)$, since $E[\textbf{a}^{\prime}\textbf{X}] = E[2X_1] = 2E[X_1]=2\mu_1=2*0 = 0$ and $V[\textbf{a}^{\prime}\textbf{X}] = V[2X_1] = 4V[X_1] = 4\sigma_1^2 = 4*1 = 4$.
    But if $X_1$ is not in the interval $-1 \leq X_1 \leq 1$, then $X_1 - X_2 = 0$. This happens with point probability

    \[
      P(
        \textbf{a}^{\prime}
        \textbf{X}
      )
      =
      P(X_1 - X_2)
      =
      P(0)
      =
      P(X_1 < -1 \text{ and } X_1 > 1)
      =
    \]
    \[
      =
      P(X_1 < -1) + P(X_1 > 1)
      =
      2P(X_1 > 1)
      =
      P(|X_1| > 1)
      =
      0.3173105
    \]
    So since $P(0) \ne 0$, $X_1$ and $X_2$ don't have a bivariate normal. Basically, a Normal distribution is either entirely continuous or entirely discrete, so here we have a discrete point mass at 0 that doesn't have a probability of zero and a continuous $N(0,4)$ distribution from -2 to 2, so this is not consistent with a normal distribution.
    \par The R code for this probability calculation is \texttt{2*pnorm(1, lower.tail=FALSE)}. Could have also done the probability calculation as
    \[
      P(|X_1| > 1)
      =
      1 - P(|X_1| \leq 1)
      =
      1 - P(-1 \leq X_1 \leq 1)
      =
    \]
    \[
      =
      1 - \left( \Phi (1) - \Phi (-1) \right)
      =
      1 - 0.6826895
      =0.3173105
    \]
    Where $\Phi$ is the CDF of a standard normal distribution. The R code is \texttt{1 - (pnorm(1)- pnorm(-1))}.
    \newline
    \textit{Hint:}
    \begin{enumerate}[label= (\alph*)]
        \item Since $X_1$ is $N(0,1)$, $P[-1 < X_1 \leq x] = P[-x \leq X_1 < 1]$ for any $x$. When $-1 < x_2 < 1$, $P[X_2 \leq x_2] = P[X_2 \leq -1] + P[-1 < X_2 \leq x_2] = P[X_1 \leq -1] + P[-1 < -X_1 \leq x_2] = P[X_1 \leq -1] + P[-x_2 \leq X_1 < 1]$. But $P[-x_2 \leq X_1 < 1] = P[-1 < X_1 \leq x_2]$ from the symmetry argument in the first line of this hint. Thus $P[X_2 \leq x_2] = P[X_1 \leq -1] + P[-1 < X_1 \leq x_2] = P[X_1 \leq x_2]$, which is a standard normal probability.
        \item Consider the linear combination $X_1 - X_2$, which equals zero with probability $P[|X_1| > 1] = 0.3174$.
    \end{enumerate}
\end{enumerate}
