\begin{enumerate}[label= (\alph*)]
    \item Evaluate $T^{2}$, for testing $H_{0}: \bm{\mu}^{\prime} = [7,\ 11]$, using the data
        \[
            \begin{bNiceArray}{cc}
                2 & 12 \\
                8 & 9  \\
                6 & 9  \\
                8 & 10
            \end{bNiceArray}
        \]

        \[
            \bar{\textbf{x}}
            =
            \begin{bNiceArray}{c}
                \bar{x}_{1} \\
                \bar{x}_{2}
            \end{bNiceArray}
            =
            \begin{bNiceArray}{c}
                \frac{2 + 8+ 6+ 8}{4} \\
                \frac{12 + 9+ 9 + 10}{4}
            \end{bNiceArray}
            =
            \begin{bNiceArray}{c}
                6 \\
                10
            \end{bNiceArray}
        \]

        \[
            s_{11}
            =
            \frac{{(2-6)}^{2} + {(8-6)}^{2} + {(6-6)}^{2} + {(8-6)}^{2}}{4 - 1}
            =
            8
        \]

        \[
            s_{12}
            =
            \frac{{(2-6)(12-10)} + {(8-6)(9-10)} + {(6-6)(9-10)} + {(8-6)(10-10)}}{4 - 1}
            =
            \frac{-10}{3}
        \]

        \[
            s_{22}
            =
            \frac{{(12-10)}^{2} + {(9-10)}^{2} + {(9-10)}^{2} + {(10-10)}^{2}}{4 - 1}
            =
            2
        \]

        \[
            \textbf{S}
            =
            \begin{bNiceArray}{cc}
                8     & -10/3 \\
                -10/3 & 2
            \end{bNiceArray}
        \]

        \[
            \textbf{S}^{-1}
            =
            \frac{1}{(8)(2) - (10/3)(10/3)}
            \begin{bNiceArray}{cc}
                2     & 10/3 \\
                10/3 & 8
            \end{bNiceArray}
            =
            \begin{bNiceArray}{cc}
                9/22  & 15/22 \\
                15/22 & 18/11
            \end{bNiceArray}
        \]

        \[
            T^{2}
            =
            n
            {\left(\bar{\textbf{x}} - \bm{\mu}\right)}^{\prime}
            S^{-1}
            \left(\bar{\textbf{x}} - \bm{\mu}\right)
            =
        \]
        \[
            =
            4
            \begin{bNiceArray}{cc}
                6 - 7, & 10 - 11
            \end{bNiceArray}
            \begin{bNiceArray}{cc}
                9/22  & 15/22 \\
                15/22 & 18/11
            \end{bNiceArray}
            \begin{bNiceArray}{c}
                6 - 7 \\
                10 - 11
            \end{bNiceArray}
            =
        \]
        \[
            =
            4
            \begin{bNiceArray}{cc}
                -1, & -1
            \end{bNiceArray}
            \begin{bNiceArray}{c}
                -24/22 \\
                -51/22
            \end{bNiceArray}
            =
            300/22
            =
            13.6364
        \]
    \item Specify the dstribution of $T^{2}$ for the situation in (a).
    \newline
    $T^{2}$ has a distribution of
        \[
            \frac{(n-1)p}{(n-p)}F_{p, n-p}(\alpha)
            =
            \frac{(4-1)2}{(4-2)}F_{2, 4-2}(\alpha)
            =
            3F_{2, 2}(\alpha)
        \]
    random variable.
    
    \item Using (a) and (b), test $H_{0}$ at the $\alpha = 0.05$ level. What conclusion do you reach.
    \[
    \begin{aligned}
        H_{0}: & \bm{\mu} = \bm{\mu}_{0} \\
        H_{1}: & \bm{\mu} \ne \bm{\mu}_{0} 
    \end{aligned}
    \]
    Comparing the observed $T^{2} = 13.6364$ with the critical value
    \[
        \frac{(n-1)p}{(n-p)}F_{p, n-p}(.05)
        =
        \frac{(3)2}{2}F_{2, 2}(.05)
        =
        3(19.00)
        =
        57
    \]
    we see that $T^{2} = 13.6364 < 57$, and consequently, we fail to reject $H_{0}$ at the 5\% level of significance.
\end{enumerate}