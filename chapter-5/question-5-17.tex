In order to assess the prevalence of a drug problem among high school students in a
particular city, a random sample of 200 students from the city's five high schools
were surveyed. One of the survey questions and the corresponding responses are
as follows:
\newline
What is your typical weekly marijuana usage?

\begin{center}
\begin{NiceTabular}{r{c}w{c}{2cm}w{c}{3cm}c}
    & \Block{1-3}{Category} \\
    \Block{2-1}{} & \Block{2-1}{None} & \Block{2-1}{Moderate \\ (1--3 joints)} & \Block{2-1}{Heavy \\ (4 or more joints)} \\
    & & & \\
    \Block{2-1}{Number of \\ responses} & \Block{2-1}{117} & \Block{2-1}{62} & \Block{2-1}{21} \\
    & & & \\
    % Draw rows.
    \CodeAfter \tikz \draw[solid] (1-|2) -- (1-|5);
    \CodeAfter \tikz \draw[solid] (2-|1) -- (2-|5);
    \CodeAfter \tikz \draw[solid] (4-|1) -- (4-|5);
    \CodeAfter \tikz \draw[solid] (6-|1) -- (6-|5);
    % Draw columns.
    \CodeAfter \tikz \draw[solid] (2-|1) -- (6-|1);
    \CodeAfter \tikz \draw[solid] (1-|2) -- (6-|2);
    \CodeAfter \tikz \draw[solid] (2-|3) -- (6-|3);
    \CodeAfter \tikz \draw[solid] (2-|4) -- (6-|4);
    \CodeAfter \tikz \draw[solid] (1-|5) -- (6-|5);
\end{NiceTabular}
\end{center}

Construct 95\% simultaneous confidence intervals for the three proportions $p_{1}$, $p_{2}$, and
$p_{3} = 1 - (p_{1} + p_{2})$,
\newline
Create the sample proportion vector
\[
    \hat{\textbf{p}}
    =
    \begin{bNiceArray}{c}
        117/200 \\
        62/200  \\
        21/200
    \end{bNiceArray}
    =
    \begin{bNiceArray}{c}
        0.585 \\
        0.310 \\
        0.105
    \end{bNiceArray}
\]
The sample covariance matrix is then
\[
    \hat{\bm{\Sigma}}
    =
    \begin{bNiceArray}{rrr}
         p_{1}(1 - p_{1}) & -p_{1}p_{2}       & -p_{1}p_{3}       \\
        -p_{1}p_{2}       &  p_{2}(1 - p_{2}) & -p_{2}p_{3}       \\
        -p_{1}p_{3}       & -p_{2}p_{3}       &  p_{3}(1 - p_{3})
    \end{bNiceArray}
    =
    \scalebox{0.85}{$
        \begin{bNiceArray}{rrr}
            0.242775 & -0.18135  & -0.061425 \\
            -0.18135  &  0.2139   & -0.03255  \\
            -0.061425 & -0.03255  &  0.093975 \\
        \end{bNiceArray}
    $}
\]
\[
    \textbf{a}^{\prime} \hat{\textbf{p}}
    \pm
    \sqrt{\chi_{q}^{2}(\alpha)}
    \sqrt{\frac{\textbf{a}^{\prime}\hat{\bm{\Sigma}}\textbf{a}}{n}}
\]
where $\chi_{3}^{2}(0.05) = 7.82$. I used 3 vectors for $\textbf{a}$,
\[
    \textbf{a}_{1}
    =
    \left[
        \begin{array}{r}
        1 \\
        0 \\
        0 \\
        0 \\
        0
        \end{array}
    \right],
    \textbf{a}_{2}
    =
    \left[
        \begin{array}{r}
        0 \\
        1 \\
        0 \\
        0 \\
        0
        \end{array}
    \right],
    \textbf{a}_{3}
        =
        \left[
            \begin{array}{r}
            0 \\
            0 \\
            1 \\
            0 \\
            0
            \end{array}
        \right],
\]

\[
    \begin{NiceArray}{rrrr}
       0.59 \pm \sqrt{7.82} \frac{\sqrt{0.24}}{\sqrt{200}} & \text{contains } p_{1} & \text{ or } & 0.49 \leq p_{1} \leq 0.68 \\
       0.31 \pm \sqrt{7.82} \frac{\sqrt{0.21}}{\sqrt{200}} & \text{contains } p_{2} & \text{ or } & 0.22 \leq p_{2} \leq 0.40 \\
       0.11 \pm \sqrt{7.82} \frac{\sqrt{0.09}}{\sqrt{200}} & \text{contains } p_{3} & \text{ or } & 0.04 \leq p_{3} \leq 0.17
    \end{NiceArray}
\]

This isn't necessary to do, but we can check that the variance for the heavy users, $p_{3}$, is the same as computing the variance of $1 - (X_{1} + X_{2})$.
\begin{align*}
    \text{Var}[X_{3}]
    &= \text{Var}[1 - (X_{1} + X_{2})] \\
    &= \text{Var}[1] + {(-1)}^{2}\text{Var}[X_{1} + X_{2}] & \text{(by independence)} \\
    &= 0 + \text{Var}[X_{1} + X_{2}] \\
    &= \text{Var}[X_{1}] +\text{Var}[X_{2}] + 2\text{Cov}[X_{1}, X_{2}] & \text{(Var of sum of $X_{1}$ and $X_{2}$)} \\
    &= p_{1}(1 - p_{1}) + p_{2}(1 - p_{2}) + 2(-p_{1}p_{2}) \\
    &= p_{1} + p_{2} - p_{1}^{2} - p_{2}^{2} - 2p_{1}p_{2} \\
    &= (p_{1} + p_{2}) - (p_{1}^{2} + p_{2}^{2} + 2p_{1}p_{2}) \\
    &= (p_{1} + p_{2}) - {(p_{1} + p_{2})}^{2} \\
    &= (1 - (p_{1} + p_{2}))(p_{1} + p_{2}) \\
    &= p_{3}(1 - p_{3})
\end{align*}