Using the data in Example 5.1, verify that $T^{2}$ remains unchanged if each observation $\textbf{x}_{j}$, $j = 1, 2, 3$, is replaced by $\textbf{C}\textbf{x}_{j}$, where
\[
    \textbf{C}
    =
    \begin{bNiceArray}{cc}
        1 & -1 \\
        1 &  1
    \end{bNiceArray}
\]
Note that the observations
\[
    \textbf{C}\textbf{x}_{j}
    =
    \begin{bNiceArray}{c}
        x_{j1} - x_{j2} \\
        x_{j1} + x_{j2}
    \end{bNiceArray}
\]
yield the data matrix
\[
    {
    \begin{bNiceArray}{ccc}
        (6-9) & (10-6) & (8-3) \\
        (6+9) & (10+6) & (8+3)
    \end{bNiceArray}
    }^{\prime}
\]

\[
    \textbf{C}\textbf{x}_{j}
    =
    \textbf{X}\textbf{C}^{\prime}
    =
    {\left(\textbf{C}\textbf{X}^{\prime}\right)}^{\prime}
    =
    {\left(
        \begin{bNiceArray}{cc}
            1 & -1 \\
            1 &  1
        \end{bNiceArray}
        \begin{bNiceArray}{ccc}
            6 & 10 & 8 \\
            9 & 6 & 3
        \end{bNiceArray}
    \right)}^{\prime}
    =
\]
\[
    =
    {
        \begin{bNiceArray}{ccc}
            (6-9) & (10-6) & (8-3) \\
            (6+9) & (10+6) & (8+3)
        \end{bNiceArray}
    }^{\prime}
    =
    {
        \begin{bNiceArray}{ccc}
            -3 & 4 & 5 \\
            15 & 16 & 11
        \end{bNiceArray}
    }^{\prime}
    =
    \begin{bNiceArray}{rr}
        -3 & 15 \\
        4  & 16 \\
        5  & 11
    \end{bNiceArray}
\]

\[
    \textbf{C}\bm{\mu}_{0}
    =
    \begin{bNiceArray}{cc}
        1 & -1 \\
        1 &  1
    \end{bNiceArray}
    \begin{bNiceArray}{c}
        9 \\
        5
    \end{bNiceArray}
    =
    \begin{bNiceArray}{c}
        4 \\
        14
    \end{bNiceArray}
\]

\[
    \bar{\textbf{x}}
    =
    \begin{bNiceArray}{c}
        \bar{x}_{1} \\
        \bar{x}_{2}
    \end{bNiceArray}
    =
    \begin{bNiceArray}{c}
        \frac{-3 + 4 + 5}{3} \\
        \frac{15 + 16 + 11}{3}
    \end{bNiceArray}
    =
    \begin{bNiceArray}{c}
        2 \\
        14
    \end{bNiceArray}
\]

\[
    s_{11}
    =
    \frac{{(-3-2)}^{2} + {(4-2)}^{2} + {(5-2)}^{2}}{3 - 1}
    =
    19
\]

\[
    s_{12}
    =
    \frac{{(-3-2)(15-14)} + {(4-2)(16-14)} + {(5-2)(11-14)}}{3 - 1}
    =
    -5
\]

\[
    s_{22}
    =
    \frac{{(15-14)}^{2} + {(16-14)}^{2} + {(11-14)}^{2}}{4 - 1}
    =
    7
\]

\[
    \textbf{S}
    =
    \begin{bNiceArray}{cc}
        19     & -5 \\
        -5 & 7
    \end{bNiceArray}
\]

\[
    \textbf{S}^{-1}
    =
    \frac{1}{(19)(7) - (-5)(-5)}
    \begin{bNiceArray}{cc}
        7 & 5  \\
        5 & 19
    \end{bNiceArray}
    =
    \begin{bNiceArray}{cc}
        7/108  & 5/108 \\
        5/108 & 19/108
    \end{bNiceArray}
\]

\[
    T^{2}
    =
    n
    {\left(\bar{\textbf{x}} - \bm{\mu}\right)}^{\prime}
    S^{-1}
    \left(\bar{\textbf{x}} - \bm{\mu}\right)
    =
\]
\[
    =
    3
    \begin{bNiceArray}{cc}
        2 - 4, & 14 - 14
    \end{bNiceArray}
    \begin{bNiceArray}{cc}
        7/108  & 5/108 \\
        5/108 & 19/108
    \end{bNiceArray}
    \begin{bNiceArray}{c}
        2 - 4 \\
        14 - 14
    \end{bNiceArray}
    =
\]
\[
    =
    3
    \begin{bNiceArray}{cc}
        -2, & 0
    \end{bNiceArray}
    \begin{bNiceArray}{c}
        -14/108 \\
        -10/108
    \end{bNiceArray}
    =
    84/108
    =
    7/9
    =
    0.7778
\]

This is the same $T^{2}$ value from Example 5.1.