Using the data on bone mineral content in Table 1.8, construct the 95\% Bonferroni
intervals for the individual means. Also, find the 95\% simultaneous $T^{2}$-intervals.
Compare the two sets of intervals.

\[
    \bar{\textbf{x}}
    =
    \begin{bNiceArray}{c}
       0.84 \\
       0.82 \\
       1.79 \\
       1.73 \\
       0.70 \\
       0.69 \\
    \end{bNiceArray}
    \hspace{0.20cm}
    \text{and}
    \hspace{0.20cm}
    \textbf{S}
    =
    \begin{bNiceArray}{cccccc}
        0.013 & 0.010 & 0.022 & 0.020 & 0.009 & 0.008 \\
        0.010 & 0.011 & 0.019 & 0.021 & 0.009 & 0.009 \\
        0.022 & 0.019 & 0.080 & 0.067 & 0.017 & 0.013 \\
        0.020 & 0.021 & 0.067 & 0.069 & 0.018 & 0.017 \\
        0.009 & 0.009 & 0.017 & 0.017 & 0.012 & 0.008 \\
        0.008 & 0.009 & 0.013 & 0.017 & 0.008 & 0.011 \\
    \end{bNiceArray}
\]
The 95\% $T^{2}$ simultaneous confidence intervals:
\[
\bar{x}_{i}
\pm
\sqrt{
    \frac{(n-1)p}{(n-p)}
    F_{p, n-p}\left(\alpha\right)
}
\sqrt{
    \frac{s_{ii}}{n}
}
\]

\[
    \begin{NiceArray}{rrrr}
       0.84 \pm \sqrt{19.92} \frac{\sqrt{0.013}}{\sqrt{25}} & \text{contains } \mu_{1} & \text{ or } & 0.74 \leq \mu_{1} \leq 0.95 \\
       0.82 \pm \sqrt{19.92} \frac{\sqrt{0.011}}{\sqrt{25}} & \text{contains } \mu_{2} & \text{ or } & 0.72 \leq \mu_{2} \leq 0.91 \\
       1.79 \pm \sqrt{19.92} \frac{\sqrt{0.080}}{\sqrt{25}} & \text{contains } \mu_{3} & \text{ or } & 1.54 \leq \mu_{3} \leq 2.05 \\
       1.73 \pm \sqrt{19.92} \frac{\sqrt{0.069}}{\sqrt{25}} & \text{contains } \mu_{4} & \text{ or } & 1.50 \leq \mu_{4} \leq 1.97 \\
       0.70 \pm \sqrt{19.92} \frac{\sqrt{0.012}}{\sqrt{25}} & \text{contains } \mu_{5} & \text{ or } & 0.61 \leq \mu_{5} \leq 0.80 \\
       0.69 \pm \sqrt{19.92} \frac{\sqrt{0.011}}{\sqrt{25}} & \text{contains } \mu_{6} & \text{ or } & 0.60 \leq \mu_{6} \leq 0.79 \\
    \end{NiceArray}
\]

The 95\% Bonferroni confidence intervals:
\[
    \bar{x}_{i}
    \pm
    t_{n-1}
    \left(\frac{\alpha}{2m}\right)
    \sqrt{
        \frac{
                s_{ii}
            }{
                n
            }
        }
\]

\[
    \begin{NiceArray}{rrrr}
       0.84 \pm 2.88 \frac{\sqrt{0.013}}{\sqrt{25}} & \text{contains } \mu_{1} & \text{ or } & 0.78 \leq \mu_{1} \leq 0.91 \\
       0.82 \pm 2.88 \frac{\sqrt{0.011}}{\sqrt{25}} & \text{contains } \mu_{2} & \text{ or } & 0.76 \leq \mu_{2} \leq 0.88 \\
       1.79 \pm 2.88 \frac{\sqrt{0.080}}{\sqrt{25}} & \text{contains } \mu_{3} & \text{ or } & 1.63 \leq \mu_{3} \leq 1.96 \\
       1.73 \pm 2.88 \frac{\sqrt{0.069}}{\sqrt{25}} & \text{contains } \mu_{4} & \text{ or } & 1.58 \leq \mu_{4} \leq 1.89 \\
       0.70 \pm 2.88 \frac{\sqrt{0.012}}{\sqrt{25}} & \text{contains } \mu_{5} & \text{ or } & 0.64 \leq \mu_{5} \leq 0.77 \\
       0.69 \pm 2.88 \frac{\sqrt{0.011}}{\sqrt{25}} & \text{contains } \mu_{6} & \text{ or } & 0.63 \leq \mu_{6} \leq 0.75 \\
    \end{NiceArray}
\]

Dividing the length of the Bonferroni interval by the length of the $T^{2}$ interval,
\[
    \frac{\text{Length of Bonferroni interval}}{\text{Length of the }T^{2}\text{-interval}}
    =
    \frac{t_{n-1}(\frac{\alpha}{2m})}{\sqrt{\frac{(n-1)p}{n-p}F_{p, n-p}(\alpha)}}
    =
    0.6442
\]
The Bonferroni interval is only 64.42\% of the length of the simultaneous $T^{2}$ interval, so it's about 35\% shorter.

