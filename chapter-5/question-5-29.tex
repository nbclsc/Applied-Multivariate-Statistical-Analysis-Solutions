Refer to the car body data in Exercise 5.28. These are all measured as deviations from
target value so it is appropriate to test the null hypothesis that the mean vector is zero.
Using the first 30 cases, test $H_{0}: \bm{\mu} = \textbf{0}$ at $\alpha = .05$
\newline
\par
Okay, so we're testing
\[
    H_{0}: \bm{\mu} = \textbf{0} = \bm{\mu}
    \hspace{0.2cm}
    \text{ versus }
    \hspace{0.2cm}
    H_{1}: \bm{\mu} \ne \textbf{0}
\]
We need two things. One to compute the $T^{2}$ value, found in (5--4) on page 211.
\[
    T^{2}
    =
    n
    {(\bar{\textbf{x}} - \bm{\mu}_{0})}^{\prime}
    \textbf{S}^{-1}
    (\bar{\textbf{x}} - \bm{\mu}_{0})
    =
    374.72
\]
The other is a critical value, computed as
\[
    \frac{(n-1)p}{n-p}
    F_{p, n-p}(\alpha)
    =
    \frac{(29)6}{24}
    (2.51)
    =
    18.18
\]

Our $T^{2}$ value of 374.72 is larger than our critical valuee of 18.18, so we'd reject the null hypothesis that $\bm{\mu} = \textbf{0}$ at the 0.05-level.