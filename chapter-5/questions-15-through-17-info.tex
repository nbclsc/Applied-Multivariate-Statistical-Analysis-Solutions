\textit{Exercises 5.15, 5.16, and 5.17 refer to the following information:}
\newline
\newline
Frequently, some or all of the population characteristics of interest are in the form of
attributes. Each individual in the population may then be described in terms of the
attributes it possesses. For convenience, attributes are usually numerically coded with respect
to their presence or absence. If we let the variable $X$ pertain to a specific attribute
then we can distinguish between the presence or absence of this attribute by defining

\[
    X
    =
    \begin{cases}
        1 & \text{if attribute present} \\
        0 & \text{if attribute absent}
    \end{cases}
\]
\newline
In this way, we can assign numerical values to qualitative characteristics.
\par
When attributes are numerically coded as $0-1$ variables, a random sample from the
population of interest results in statistics that consist of the counts of the number of
sample items that have each distinct set of characteristics. If the sample counts are
large, methods for producing simultaneous confidence statements can be easily adapted
to situations involving proportions.
\par
We consider the situation where an individual with a particular combination of
attributes can be classified into one of q + 1 mutually exclusive and exhaustive
categories. The corresponding probabilities are denoted by $p_{1}, p_{2}, \dots, p_{q}, p_{q+1}$. Since
the categories include all possibilities, we take $p_{q+1} = 1 - (p_{1} + p_{2} + \dots + p_{q})$, An
individual from category k will be assigned the $( ( q + 1) \times 1 )$ vector value ${[ 0, \dots , 0, 1, 0, \dots , 0]}^{\prime}$ with 1 in the $k$th position.
\par
The probability distribution for an observation from the population of individuals in
$q + 1$ mutually exclusive and exhaustive categories is known as the multinomial distribution.
It has the following structure:

\begin{center}
    \begin{tabular}{p{5em} ccccccc}
        Category & 1 & 2 & $\cdots$ & $k$ & $\cdots$ & $q$ & $q + 1$ \\
        Outcome (value) &

        $\begin{bNiceArray}{c}
            1 \vphantom{\vdots} \\
            0 \vphantom{\vdots} \\
            0 \vphantom{\vdots} \\
            \vdots \\
            \vdots \\
            \vdots \\
            0 \vphantom{\vdots}
        \end{bNiceArray}$ &

        $\begin{bNiceArray}{c}
            0 \vphantom{\vdots} \\
            1 \vphantom{\vdots} \\
            0 \vphantom{\vdots} \\
            \vdots \\
            \vdots \\
            \vdots \\
            0 \vphantom{\vdots}
        \end{bNiceArray}$ &

        $\cdots$ &

        $\begin{bNiceArray}{c}
            0 \vphantom{\vdots} \\
            \vdots \\
            0 \vphantom{\vdots} \\
            1 \vphantom{\vdots} \\
            0 \vphantom{\vdots} \\
            \vdots \\
            0 \vphantom{\vdots}
        \end{bNiceArray}$ &

        $\cdots$ &

        $\begin{bNiceArray}{c}
            0 \vphantom{\vdots} \\
            0 \vphantom{\vdots} \\
            0 \vphantom{\vdots} \\
            \vdots \\
            0 \vphantom{\vdots} \\
            1 \vphantom{\vdots} \\
            0 \vphantom{\vdots}
        \end{bNiceArray}$ &
       
        $\begin{bNiceArray}{c}
            0 \vphantom{\vdots} \\
            0 \vphantom{\vdots} \\
            0 \vphantom{\vdots} \\
            \vdots \\
            \vdots \\
            0 \vphantom{\vdots} \\
            1 \vphantom{\vdots}
        \end{bNiceArray}$ \\
        probability \\ (proportion) & $p_{1}$ & $p_{2}$ & $\cdots$ & $p_{k}$ & $\cdots$ & $p_{q}$ & $p_{q+1} = 1 - \sum_{i=1}^{q}{p_{i}}$ 
    \end{tabular}
\end{center}

Let $\textbf{X}_{j}$, $j = 1, 2, \dots, n$, be a random sample of size $n$ from the multinomial distribution.
\par
The kth component, $X_{jk}$ of $\textbf{X}_{j}$ is 1 if the observation (individual) is from category $k$ and is 0 otheiwise.
The random sample $\textbf{X}_{1}, \textbf{X}_{2}, \dots, \textbf{X}_{n}$ can be converted to a sample proportion vector, which, given the nature of the preceding observations, is a sample
mean vector.
Thus,

\[
    \hat{\textbf{P}}
    =
    \begin{bNiceArray}{c}
        \hat{p}_{1} \\
        \hat{p}_{2} \\
        \vdots \\
        \hat{p}_{q+1}
    \end{bNiceArray}
    =
    \frac{1}{n}
    \sum_{j=1}^{n}{\textbf{X}_{j}}
    \hspace{0.5cm}
    \text{with}
    \hspace{0.5cm}
    E(\hat{\textbf{p}})
    =
    \textbf{p}
    =
    \begin{bNiceArray}{c}
        p_{1} \\
        p_{2} \\
        \vdots \\
        p_{q+1}
    \end{bNiceArray}
\]
and
\[
    \text{Cov}(\hat{\textbf{p}})
    =
    \frac{1}{n}
    \text{Cov}(\textbf{X}_{j})
    =
    \frac{1}{n}
    \bm{\Sigma}
    =
    \frac{1}{n}
    \begin{bNiceArray}{cccc}
        \sigma_{11} & \sigma_{12} & \cdots & \sigma_{1, q+1} \\
        \sigma_{21} & \sigma_{22} & \cdots & \sigma_{2, q+1} \\
        \vdots      & \vdots      & \ddots & \vdots \\
        \sigma_{1, q+1} & \sigma_{2, q+1} & \cdots & \sigma_{q+1, q+1} \\
    \end{bNiceArray}
\]

For large $n$, the approximate sampling distribution of $\hat{\textbf{p}}$ is provided by the central limit theorem. We have

\[
    \sqrt{n} \left(\hat{\textbf{p}} - \textbf{p}\right)
    \hspace{0.5cm}
    \text{is approximately}
    \hspace{0.5cm}
    N (\textbf{0}, \bm{\Sigma})
\]

where the elements of $\bm{\Sigma}$ are $\sigma_{kk} = p_{k}( 1 - p_{k})$ and $\sigma_{ik} = -p_{i}p_{k}$. The normal approximation
remains valid when $\sigma_{kk}$ is estimated by $\hat{\sigma}_{kk} = \hat{p}_{k}(1 - \hat{p}_{k})$ and $\sigma_{ik}$ is estimated
by $\sigma_{ik} = -\hat{p}_{i}\hat{p}_{k}$, $i \ne k$.
\par
Since each individual must belong to exactly one category, $X_{q+1,i} =
1 - (X_{1i} + X_{2i} + \dots + X_{qj})$, so $\hat{p}_{q+1} = 1 - (\hat{p}_{1} + \hat{p}_{2} + \dots + \hat{p}_{q})$, and as a result, $\hat{\bm{\Sigma}}$
has rank $q$.
The usual inverse of $\hat{\bm{\Sigma}}$ does not exist, but it is still possible to develop simultaneous
$100(1 - \alpha)$\% confidence intervals for all linear combinations $\textbf{a}^{\prime}\textbf{p}$.
\newline
\newline
\textbf{Result.} Let $\textbf{X}_{1} , \textbf{X}_{2} , \dots, \text{X}_{n}$ be a random sample from a $q + 1$ category multinoinial
distribution with $P[X_{jk} = 1] = p_{k}$, $k = 1, 2, \dots, q + 1$, $j = 1, 2, \dots, n$.
Approximate
simultaneous $100(1 - \alpha)$\% confidence regions for all linear combinations $\textbf{a}^{\prime}\textbf{p} = a_{1}p_{1} + a_{2}p_{2} + \dots + a_{q+1}p_{q+1}$ are given by the observed values of

\[
    \textbf{a}^{\prime}\textbf{p}
    \pm
    \sqrt{\chi^{2}_{q}(\alpha)}
    \sqrt{\frac{\textbf{a}^{\prime}\hat{\bm{\Sigma}}\textbf{a}}{n}}
\]

provided that $n-q$ is large. Here $p = (1/n) \sum_{j=1}^{n}{\textbf{X}_{j}}$, and $\hat{\bm{\Sigma}} = {\hat{\sigma}_{ik}}$ is a $(q + 1) \times (q + 1)$
matrix with $\hat{\sigma}_{kk} = \hat{p}_{k}(l - \hat{p}_{k})$ and $\hat{\sigma}_{ik} = -\hat{p}_{i}\hat{p}_{k}$, $i \ne k$. Also, $\chi^{2}_{q}(\alpha)$ is the upper
$(l00\alpha)$th percentile of the chi-square distribution with $q$ d.f.
\newline
\newline
\par
In this result, the requirement that $n - q$ is large is interpreted to mean $n\hat{p}_{k}$ is about 20 or more for each category.
\par
We have only touched on the possibilities for the analysis of categorical data. Complete discussions of categorical data analysis are available in [l] and [4].