\textit{(Two-way MANOVA without replications.)} Consider the observations on two
responses, $x_{1}$ and $x_{2}$, displayed in the form of the following two-way table (note that
there is a single observation vector at each combination of factor levels):

\begin{center}
    \begin{NiceTabular}{cccccc}[margin,cell-space-limits=1.5mm]
        & & \Block{1-4}{Factor 2} \\
        &
        &
        \Block[c]{}{Level \\ 1}
        &
        \Block[c]{}{Level \\ 2}
        &
        \Block[c]{}{Level \\ 3}
        &
        \Block[c]{}{Level \\ 4} \\
        \Block{3-1}{Factor 1} & Level 1 &
        $\left[
            \begin{array}{r}
                6 \\
                8
            \end{array}
        \right]$
        &
        $\left[
            \begin{array}{r}
                4 \\
                6
            \end{array}
        \right]$
        &
        $\left[
            \begin{array}{r}
                8 \\
                12
            \end{array}
        \right]$
        &
        $\left[
            \begin{array}{r}
                2 \\
                6
            \end{array}
        \right]$
        \\
        & Level 2 &
        $\left[
            \begin{array}{r}
                3 \\
                8
            \end{array}
        \right]$
        &
        $\left[
            \begin{array}{r}
                -3 \\
                2
            \end{array}
        \right]$
        &
        $
        \left[
            \begin{array}{r}
                4 \\
                3
            \end{array}
        \right]$
        &
        $\left[
            \begin{array}{r}
                -4 \\
                3
            \end{array}
        \right]$
        \\
        & Level 3 &
        $\left[
            \begin{array}{r}
                -3 \\
                2
            \end{array}
        \right]$
        &
        $\left[
            \begin{array}{r}
                -4 \\
                -5
            \end{array}
        \right]$
        &
        $\left[
            \begin{array}{r}
                3 \\
                -3
            \end{array}
        \right]$
        &
        $\left[
            \begin{array}{r}
                -4 \\
                -6
            \end{array}
        \right]$
        \CodeAfter \tikz \draw[solid] (1-|1) -- (1-|last);
                \tikz \draw[solid] (3-|1) -- (3-|last);
                \tikz \draw[solid] (last-|1) -- (last-|last);
                \tikz \draw[solid] (1-|3) -- (last-|3);
    \end{NiceTabular}
\end{center}
With no replications, the two-way MANOVA model is
\[
    \textbf{X}_{\ell k}
    =
    \bm{\mu}
    +
    \bm{\tau}_{\ell}
    +
    \bm{\beta}_{k}
    +
    \textbf{e}_{\ell k}
    \text{;}
    \hspace{0.75cm}
    \sum_{\ell = 1}^{\text{g}}
    \bm{\tau}_{\ell}
    =
    \sum_{k = 1}^{b}
    \bm{\beta}_{k}
    =
    \textbf{0}
\]
where the $\textbf{e}_{\ell k}$ are independent $N_{p}(\textbf{0}, \bm{\Sigma})$ random vectors.
\begin{enumerate}[label= (\alph*)]
    \item Decompose the observations for each of the two variables as
    \[
        x_{\ell k}
        =
        \bar{x}
        +
        (
            \bar{x}_{\ell \cdot}
            -
            \bar{x}
        )
        +
        (
            \bar{x}_{\cdot k}
            -
            \bar{x}
        )
        +
        (
            x_{\ell k}
            -
            \bar{x}_{\ell \cdot}
            -
            \bar{x}_{\cdot k}
            +
            \bar{x}
        )
    \]
    similar to the arrays in Example 6.9. \textit{For each response}, this decomposition will result
in several $3 \times 4$ matrices.Here $\bar{x}$ is the overall average, $\bar{x}_{\ell \cdot}$ is the average for the $\ell$th
level of factor 1, and $\bar{x}_{\cdot k}$ is the average for the $k$th level of factor 2.

\underline{Variable $x_{1}$:}
    \begin{multline*}
        \underset{\text{(observation)}}{
            \left[
                \begin{array}{rrrr}
                    6 &  4 & 8 &  2 \\
                    3 & -3 & 4 & -4 \\
                    -3 & -4 & 3 & -4
                \end{array}
            \right]
        }
        =
        \underset{\text{(mean)}}{
            \left[
                \begin{array}{rrrr}
                    1 & 1 & 1 & 1 \\
                    1 & 1 & 1 & 1 \\
                    1 & 1 & 1 & 1
                \end{array}
            \right]
        }
        +
        \underset{\text{(treatment 1 effect)}}{
            \left[
                \begin{array}{rrrr}
                    4 &  4 &  4 &  4 \\
                    -1 & -1 & -1 & -1 \\
                    -3 & -3 & -3 & -3
                \end{array}
            \right]
        }
        \\
        +
        \underset{\text{(treatment 2 effect)}}{
            \left[
                \begin{array}{rrrr}
                    1 & -2 & 4 & -3 \\
                    1 & -2 & 4 & -3 \\
                    1 & -2 & 4 & -3
                \end{array}
            \right]
        }
        +
        \underset{\text{(residual)}}{
            \left[
                \begin{array}{rrrr}
                    0 &  1 & -1 &  0 \\
                    2 & -1 &  0 & -1 \\
                    -2 &  0 &  1 &  1
                \end{array}
            \right]
        }
    \end{multline*}


    \underline{Variable $x_{2}$:}
    \begin{multline*}
        \underset{\text{(observation)}}{
            \left[
                \begin{array}{rrrr}
                    8 &  6 & 12 &  6 \\
                    8 &  2 &  3 &  3 \\
                    2 & -5 & -3 & -6
                \end{array}
            \right]
        }
        =
        \underset{\text{(mean)}}{
            \left[
                \begin{array}{rrrr}
                    3 & 3 & 3 & 3 \\
                    3 & 3 & 3 & 3 \\
                    3 & 3 & 3 & 3
                \end{array}
            \right]
        }
        +
        \underset{\text{(treatment 1 effect)}}{
            \left[
                \begin{array}{rrrr}
                     5 &  5 &  5 &  5 \\
                     1 &  1 &  1 &  1 \\
                    -6 & -6 & -6 & -6
                \end{array}
            \right]
        }
        \\
        +
        \underset{\text{(treatment 2 effect)}}{
            \left[
                \begin{array}{rrrr}
                    3 & -2 & 1 & -2 \\
                    3 & -2 & 1 & -2 \\
                    3 & -2 & 1 & -2
                \end{array}
            \right]
        }
        +
        \underset{\text{(residual)}}{
            \left[
                \begin{array}{rrrr}
                    -3 &  0 &  3 &  0 \\
                     1 &  0 & -2 &  1 \\
                     2 &  0 & -1 & -1
                \end{array}
            \right]
        }
    \end{multline*}

    \item Regard the rows of the matrices in Part a as strung out in a single ``long'' vector, and
    compute the sums of squares
    \[
        \text{SS}_{\text{tot}}
        =
        \text{SS}_{\text{mean}}
        +
        \text{SS}_{\text{fac 1}}
        +
        \text{SS}_{\text{fac 2}}
        +
        \text{SS}_{\text{res}}
    \]
    and sums of cross products
    \[
        \text{SCP}_{\text{tot}}
        =
        \text{SCP}_{\text{mean}}
        +
        \text{SCP}_{\text{fac 1}}
        +
        \text{SCP}_{\text{fac 2}}
        +
        \text{SCP}_{\text{res}}
    \]
    Consequently, obtain the matrices $\textbf{SSP}_{\text{cor}}$,
    $\textbf{SSP}_{\text{fac 1}}$,
    $\textbf{SSP}_{\text{fac 2}}$,
    and \newline $\textbf{SSP}_{\text{res}}$
    with degrees of freedom $\text{g}b - 1$, $\text{g} - 1$, $b - 1$, and $(\text{g} - 1)(b - 1)$, respectively.

    What they're saying here is to apply the Vec operator to the matrices and take the dot product to complete the computations, but instead, I'm just going use the formula summarized in the solution to Exercise 6.8 to do this.
    That is applying the formula below to our matrices
    \[
        \left[
            \begin{array}{cc}
                \text{sum}(\textbf{A}_{1} \circ \textbf{A}_{1}) & \text{sum}(\textbf{A}_{1} \circ \textbf{A}_{2}) \\
                \text{sum}(\textbf{A}_{2} \circ \textbf{A}_{1}) & \text{sum}(\textbf{A}_{2} \circ \textbf{A}_{2})
            \end{array}
        \right]
        =
        \left[
            \begin{array}{cc}
                \text{tr}(\textbf{A}_{1} \textbf{A}_{1}^{\prime}) & \text{tr}(\textbf{A}_{1} \textbf{A}_{2}^{\prime}) \\
                \text{tr}(\textbf{A}_{2} \textbf{A}_{1}^{\prime}) & \text{tr}(\textbf{A}_{2} \textbf{A}_{2}^{\prime})
            \end{array}
        \right]
    \]

    \begin{align*}
        \text{tr}(\textbf{T}_{11} \textbf{T}_{11}^{\prime})
        & =
        \text{tr}
        \left(
            \begin{bNiceArray}{rrrr}
                 4 &  4 &  4 &  4 \\
                -1 & -1 & -1 & -1 \\
                -3 & -3 & -3 & -3
            \end{bNiceArray}
            \begin{bNiceArray}{rrr}
                4 & -1 & -3 \\
                4 & -1 & -3 \\
                4 & -1 & -3 \\
                4 & -1 & -3
            \end{bNiceArray}
    \right) \\
    & =
    \text{tr}
        \left(
            \begin{bNiceArray}{rrr}
                 64 & -16 & -48 \\
                -16 &   4 &  12 \\
                -48 &  12 &  36
            \end{bNiceArray}
        \right) \\
        & =
        104
    \end{align*}

    \begin{align*}
        \text{tr}(\textbf{T}_{11} \textbf{T}_{21}^{\prime})
        & =
        \text{tr}(\textbf{T}_{21} \textbf{T}_{11}^{\prime}) \\
        & =
        \text{tr}
        \left(
            \begin{bNiceArray}{rrrr}
                4 &  4 &  4 &  4 \\
               -1 & -1 & -1 & -1 \\
               -3 & -3 & -3 & -3
           \end{bNiceArray}
           \begin{bNiceArray}{rrr}
               5 & 1 & -6 \\
               5 & 1 & -6 \\
               5 & 1 & -6 \\
               5 & 1 & -6
           \end{bNiceArray}
    \right) \\
    & =
    \text{tr}
        \left(
            \begin{bNiceArray}{rrr}
                 80 &  16 & -96 \\
                -20 &  -4 &  24 \\
                -60 & -12 &  72
            \end{bNiceArray}
        \right) \\
        & =
        148
    \end{align*}

    \begin{align*}
        \text{tr}(\textbf{T}_{21} \textbf{T}_{21}^{\prime})
        & =
        \text{tr}
        \left(
            \begin{bNiceArray}{rrrr}
                5 &  5 &  5 &  5 \\
                1 &  1 &  1 &  1 \\
               -6 & -6 & -6 & -6
           \end{bNiceArray}
           \begin{bNiceArray}{rrr}
               5 & 1 & -6 \\
               5 & 1 & -6 \\
               5 & 1 & -6 \\
               5 & 1 & -6
           \end{bNiceArray}
    \right) \\
    & =
    \text{tr}
        \left(
            \begin{bNiceArray}{rrr}
                 100 &  20 & -120 \\
                  20 &   4 &  -24 \\
                -120 & -24 &  144
            \end{bNiceArray}
        \right) \\
        & =
        248
    \end{align*}

    \[
        \textbf{SSP}_{\text{fac 1}}
        =
        \textbf{B}_{1}
        =
        \begin{bNiceArray}{cc}
            \text{tr}(\textbf{T}_{11} \textbf{T}_{11}^{\prime}) & \text{tr}(\textbf{T}_{11} \textbf{T}_{21}^{\prime}) \\
            \text{tr}(\textbf{T}_{21} \textbf{T}_{11}^{\prime}) & \text{tr}(\textbf{T}_{21} \textbf{T}_{21}^{\prime})
        \end{bNiceArray}
        =
        \begin{bNiceArray}{cc}
            104 & 148 \\
            148 & 248
        \end{bNiceArray}
    \]


    \begin{align*}
        \text{tr}(\textbf{T}_{12} \textbf{T}_{12}^{\prime})
        & =
        \text{tr}
        \left(
            \begin{bNiceArray}{rrrr}
                1 & -2 & 4 & -3 \\
                1 & -2 & 4 & -3 \\
                1 & -2 & 4 & -3
            \end{bNiceArray}
            \begin{bNiceArray}{rrr}
                 1 &  1 &  1 \\
                -2 & -2 & -2 \\
                 4 &  4 &  4 \\
                -3 & -3 & -3
            \end{bNiceArray}
    \right) \\
    & =
    \text{tr}
        \left(
            \begin{bNiceArray}{rrr}
                30 & 30 & 30 \\
                30 & 30 & 30 \\
                30 & 30 & 30
            \end{bNiceArray}
        \right) \\
        & =
        90
    \end{align*}

    \begin{align*}
        \text{tr}(\textbf{T}_{12} \textbf{T}_{22}^{\prime})
        & =
        \text{tr}(\textbf{T}_{22} \textbf{T}_{12}^{\prime}) \\
        & =
        \text{tr}
        \left(
            \begin{bNiceArray}{rrrr}
                1 & -2 & 4 & -3 \\
                1 & -2 & 4 & -3 \\
                1 & -2 & 4 & -3
            \end{bNiceArray}
            \begin{bNiceArray}{rrr}
                 3 &  3 &  3 \\
                -2 & -2 & -2 \\
                 1 &  1 &  1 \\
                -2 & -2 & -2
            \end{bNiceArray}
    \right) \\
    & =
    \text{tr}
        \left(
            \begin{bNiceArray}{rrr}
                17 & 17 & 17 \\
                17 & 17 & 17 \\
                17 & 17 & 17
            \end{bNiceArray}
        \right) \\
        & =
        51
    \end{align*}

    \begin{align*}
        \text{tr}(\textbf{T}_{22} \textbf{T}_{22}^{\prime})
        & =
        \text{tr}
        \left(
            \begin{bNiceArray}{rrrr}
                3 & -2 & 1 & -2 \\
                3 & -2 & 1 & -2 \\
                3 & -2 & 1 & -2
            \end{bNiceArray}
            \begin{bNiceArray}{rrr}
                 3 &  3 &  3 \\
                -2 & -2 & -2 \\
                 1 &  1 &  1 \\
                -2 & -2 & -2
            \end{bNiceArray}
    \right) \\
    & =
    \text{tr}
        \left(
            \begin{bNiceArray}{rrr}
                18 & 18 & 18 \\
                18 & 18 & 18 \\
                18 & 18 & 18
            \end{bNiceArray}
        \right) \\
        & =
        54
    \end{align*}

    \[
        \textbf{SSP}_{\text{fac 2}}
        =
        \textbf{B}_{2}
        =
        \begin{bNiceArray}{cc}
            \text{tr}(\textbf{T}_{12} \textbf{T}_{12}^{\prime}) & \text{tr}(\textbf{T}_{12} \textbf{T}_{22}^{\prime}) \\
            \text{tr}(\textbf{T}_{22} \textbf{T}_{12}^{\prime}) & \text{tr}(\textbf{T}_{22} \textbf{T}_{22}^{\prime})
        \end{bNiceArray}
        =
        \begin{bNiceArray}{cc}
            90 & 51 \\
            51 & 54
        \end{bNiceArray}
    \]

    \begin{align*}
        \text{tr}(\textbf{E}_{1} \textbf{E}_{1}^{\prime})
        & =
        \text{tr}
        \left(
            \begin{bNiceArray}{rrrr}
                 0 &  1 & -1 &  0 \\
                 2 & -1 &  0 & -1 \\
                -2 &  0 &  1 &  1
            \end{bNiceArray}
            \begin{bNiceArray}{rrr}
                 0 &  2 & -2 \\
                 1 & -1 &  0 \\
                -1 &  0 &  1 \\
                 0 & -1 &  1
            \end{bNiceArray}
    \right) \\
    & =
    \text{tr}
        \left(
            \begin{bNiceArray}{rrr}
                 2 & -1 & -1 \\
                -1 &  6 & -5 \\
                -1 & -5 &  6
            \end{bNiceArray}
        \right) \\
        & =
        14
    \end{align*}

    \begin{align*}
        \text{tr}(\textbf{E}_{1} \textbf{E}_{2}^{\prime})
        & =
        \text{tr}(\textbf{E}_{2} \textbf{E}_{1}^{\prime}) \\
        & =
        \text{tr}
        \left(
            \begin{bNiceArray}{rrrr}
                0 &  1 & -1 &  0 \\
                2 & -1 &  0 & -1 \\
               -2 &  0 &  1 &  1
           \end{bNiceArray}
           \begin{bNiceArray}{rrr}
                -3 &  1 &  2 \\
                 0 &  0 &  0 \\
                 3 & -2 & -1 \\
                 0 &  1 & -1
           \end{bNiceArray}
    \right) \\
    & =
    \text{tr}
        \left(
            \begin{bNiceArray}{rrr}
                -3 &  2 &  1 \\
                -6 &  1 &  5 \\
                 9 & -3 & -6
            \end{bNiceArray}
        \right) \\
        & =
        -8
    \end{align*}

    \begin{align*}
        \text{tr}(\textbf{E}_{2} \textbf{E}_{2}^{\prime})
        & =
        \text{tr}
        \left(
            \begin{bNiceArray}{rrrr}
                -3 & 0 &  3 &  0 \\
                 1 & 0 & -2 &  1 \\
                 2 & 0 & -1 & -1
           \end{bNiceArray}
           \begin{bNiceArray}{rrr}
                -3 &  1 &  2 \\
                 0 &  0 &  0 \\
                 3 & -2 & -1 \\
                 0 &  1 & -1
           \end{bNiceArray}
    \right) \\
    & =
    \text{tr}
        \left(
            \begin{bNiceArray}{rrr}
                18 & -9 & -9 \\
                -9 &  6 &  3 \\
                -9 &  3 &  6
            \end{bNiceArray}
        \right) \\
        & =
        30
    \end{align*}

    \[
        \textbf{SSP}_{\text{res}}
        =
        \textbf{W}
        =
        \begin{bNiceArray}{cc}
            \text{tr}(\textbf{E}_{1} \textbf{E}_{1}^{\prime}) & \text{tr}(\textbf{E}_{1} \textbf{E}_{2}^{\prime}) \\
            \text{tr}(\textbf{E}_{2} \textbf{E}_{1}^{\prime}) & \text{tr}(\textbf{E}_{2} \textbf{E}_{2}^{\prime})
        \end{bNiceArray}
        =
        \begin{bNiceArray}{rr}
            14 & -8 \\
            -8 & 30
        \end{bNiceArray}
    \]


To compute the $\textbf{SSP}_{\text{cor}}$, we need the observation and mean sum of squares and sum of cross-product matrices.
The observation matrix is computed as
    \begin{align*}
        \text{tr}(\textbf{O}_{1} \textbf{O}_{1}^{\prime})
        & =
        \text{tr}
        \left(
            \begin{bNiceArray}{rrrr}
                 6 &  4 & 8 &  2 \\
                 3 & -3 & 4 & -4 \\
                -3 & -4 & 3 & -4
            \end{bNiceArray}
            \begin{bNiceArray}{rrr}
                6 &  3 & -3 \\
                4 & -3 & -4 \\
                8 &  4 &  3 \\
                2 & -4 &  4
            \end{bNiceArray}
    \right) \\
    & =
    \text{tr}
        \left(
            \begin{bNiceArray}{rrr}
                120 & 30 & -18 \\
                 30 & 50 &  31 \\
                -18 & 31 &  50
            \end{bNiceArray}
        \right) \\
        & =
        220
    \end{align*}

    \begin{align*}
        \text{tr}(\textbf{O}_{1} \textbf{O}_{2}^{\prime})
        & =
        \text{tr}(\textbf{O}_{2} \textbf{O}_{1}^{\prime}) \\
        & =
        \text{tr}
        \left(
            \begin{bNiceArray}{rrrr}
                6 &  4 & 8 &  2 \\
                3 & -3 & 4 & -4 \\
               -3 & -4 & 3 & -4
           \end{bNiceArray}
           \begin{bNiceArray}{rrr}
                8 & 8 &  2 \\
                6 & 2 & -5 \\
               12 & 3 & -3 \\
                6 & 3 & -6
           \end{bNiceArray}
    \right) \\
    & =
    \text{tr}
        \left(
            \begin{bNiceArray}{rrr}
                180 &  86 & -44 \\
                 30 &  18 &  33 \\
                -36 & -35 &  29
            \end{bNiceArray}
        \right) \\
        & =
        227
    \end{align*}

    \begin{align*}
        \text{tr}(\textbf{O}_{2} \textbf{O}_{2}^{\prime})
        & =
        \text{tr}
        \left(
            \begin{bNiceArray}{rrrr}
                8 &  6 & 12 &  6 \\
                8 &  2 &  3 &  3 \\
                2 & -5 & -3 & -6
           \end{bNiceArray}
           \begin{bNiceArray}{rrr}
                8 & 8 &  2 \\
                6 & 2 & -5 \\
               12 & 3 & -3 \\
                6 & 3 & -6
           \end{bNiceArray}
    \right) \\
    & =
    \text{tr}
        \left(
            \begin{bNiceArray}{rrr}
                280 & 130 & -86 \\
                130 &  86 & -21 \\
                -86 & -21 &  74
            \end{bNiceArray}
        \right) \\
        & =
        440
    \end{align*}

    \[
        \textbf{SSP}_{\text{obs}}
        =
        \begin{bNiceArray}{cc}
            \text{tr}(\textbf{O}_{1} \textbf{O}_{1}^{\prime}) & \text{tr}(\textbf{O}_{1} \textbf{O}_{2}^{\prime}) \\
            \text{tr}(\textbf{O}_{2} \textbf{O}_{1}^{\prime}) & \text{tr}(\textbf{O}_{2} \textbf{O}_{2}^{\prime})
        \end{bNiceArray}
        =
        \begin{bNiceArray}{rr}
            220 & 219 \\
            219 & 440
        \end{bNiceArray}
    \]
    The mean matrix is computed as
    \begin{align*}
        \text{tr}(\textbf{M}_{1} \textbf{M}_{1}^{\prime})
        & =
        \text{tr}
        \left(
            \begin{bNiceArray}{rrrr}
                1 & 1 & 1 & 1 \\
                1 & 1 & 1 & 1 \\
                1 & 1 & 1 & 1
            \end{bNiceArray}
            \begin{bNiceArray}{rrr}
                1 & 1 & 1 \\
                1 & 1 & 1 \\
                1 & 1 & 1 \\
                1 & 1 & 1
            \end{bNiceArray}
    \right) \\
    & =
    \text{tr}
        \left(
            \begin{bNiceArray}{rrr}
                4 & 4 & 4 \\
                4 & 4 & 4 \\
                4 & 4 & 4
            \end{bNiceArray}
        \right) \\
        & =
        12
    \end{align*}

    \begin{align*}
        \text{tr}(\textbf{M}_{1} \textbf{M}_{2}^{\prime})
        & =
        \text{tr}(\textbf{M}_{2} \textbf{M}_{1}^{\prime}) \\
        & =
        \text{tr}
        \left(
            \begin{bNiceArray}{rrrr}
                1 & 1 & 1 & 1 \\
                1 & 1 & 1 & 1 \\
                1 & 1 & 1 & 1
            \end{bNiceArray}
            \begin{bNiceArray}{rrr}
                3 & 3 & 3 \\
                3 & 3 & 3 \\
                3 & 3 & 3 \\
                3 & 3 & 3
            \end{bNiceArray}
    \right) \\
    & =
    \text{tr}
        \left(
            \begin{bNiceArray}{rrr}
                12 & 12 & 12 \\
                12 & 12 & 12 \\
                12 & 12 & 12
            \end{bNiceArray}
        \right) \\
        & =
        36
    \end{align*}

    \begin{align*}
        \text{tr}(\textbf{M}_{2} \textbf{M}_{2}^{\prime})
        & =
        \text{tr}
        \left(
            \begin{bNiceArray}{rrrr}
                3 & 3 & 3 & 3 \\
                3 & 3 & 3 & 3 \\
                3 & 3 & 3 & 3
            \end{bNiceArray}
            \begin{bNiceArray}{rrr}
                3 & 3 & 3 \\
                3 & 3 & 3 \\
                3 & 3 & 3 \\
                3 & 3 & 3
            \end{bNiceArray}
    \right) \\
    & =
    \text{tr}
        \left(
            \begin{bNiceArray}{rrr}
                36 & 36 & 36 \\
                36 & 36 & 36 \\
                36 & 36 & 36
            \end{bNiceArray}
        \right) \\
        & =
        108
    \end{align*}

    \[
        \textbf{SSP}_{\text{mean}}
        =
        \begin{bNiceArray}{cc}
            \text{tr}(\textbf{M}_{1} \textbf{M}_{1}^{\prime}) & \text{tr}(\textbf{M}_{1} \textbf{M}_{2}^{\prime}) \\
            \text{tr}(\textbf{M}_{2} \textbf{M}_{1}^{\prime}) & \text{tr}(\textbf{M}_{2} \textbf{M}_{2}^{\prime})
        \end{bNiceArray}
        =
        \begin{bNiceArray}{rr}
            12 & 36  \\
            36 & 108
        \end{bNiceArray}
    \]
    \begin{align*}
        \textbf{SSP}_{\text{cor}}
        & =
        \textbf{SSP}_{\text{obs}}
        -
        \textbf{SSP}_{\text{mean}}
        \\
        & =
        \begin{bNiceArray}{rr}
            220 & 219 \\
            219 & 440
        \end{bNiceArray}
        -
        \begin{bNiceArray}{rr}
            12 & 36  \\
            36 & 108
        \end{bNiceArray}
        \\
        & =
        \begin{bNiceArray}{rr}
            208 & 227 \\
            227 & 332
        \end{bNiceArray}
    \end{align*}

    \item Summarize the calcualtions in Part b in a MANOVA table.
    
    \textit{Hint:} This MANOVA table is consistent with the two-way MANOVA table for comparing
factors and their interactions where $n = 1$. Note that, with $n = 1$, $\textbf{SSP}_{\text{res}}$ in the
general two-way MANOVA table is a zero matrix with zero degrees of freedom. The
matrix of interaction sum of squares and cross products now becomes the residual sum
of squares and cross products matrix.

\[
    \begin{array}{lll}
        \text{Source} & \text{Matrix of sum of squares} &  \\
        \text{of variation} & \text{and cross products} & \text{Degrees of freedom} \\
        \hline \\
        \text{Treatment 1} & 
        \left[
            \begin{array}{rr}
                104 & 148 \\
                148 & 248
            \end{array}
        \right] & 
        g - 1 = 3 - 1 = 2 \\ \\
        \text{Treatment 2} & 
        \left[
            \begin{array}{rr}
                90 & 51 \\
                51 & 54
            \end{array}
        \right] & 
        b - 1 = 4 - 1 = 3 \\ \\
        \text{Residual} & 
        \left[
            \begin{array}{rr}
                14 & -8 \\
                -8 & 30
            \end{array}
        \right] &
        (g - 1)(b - 1) = 6 \\ \\
        \hline \\
        \text{Total (corrected)} & 
        \left[
            \begin{array}{rr}
                208 & 227 \\
                227 & 332
            \end{array}
        \right] & 
        gb - 1 = 11
        \end{array}
\]

    \item Given the summary in Part c, test for factor 1 and factor 2 main effects at the $\alpha = .05$ level.
    
    \textit{Hint:} Use the results in (6--67) and (6--69) with $\text{g}b(n - 1)$ replaced by $(\text{g} - 1)(b - 1)$.

    \textit{Note:} The tests require that $p \leq (\text{g} - 1) (b - 1)$ so that $\textbf{SSP}_{\text{res}}$ will be positive definite
    (with probability 1).

    \[
        \Lambda_{\text{fac 1}}^{\star}
        =
        \frac{\left| \text{SSP}_{\text{res}} \right|}{\left| \text{SSP}_{\text{fac 1}} + \text{SSP}_{\text{res}} \right|}
        =
        \frac{356}{13204}
        =
        0.0270
    \]

    \[
        X_{\text{test (fac 1)}}^{2}
        =
        -\left[(g-1)(b-1) - \frac{p+1-(g-1)}{2}\right]
        \ln \Lambda_{\text{fac 1}}^{\star}
        =
        19.8734
    \]
    
    \[
        X_{\text{crit (fac 1)}}^{2}
        =
        \chi_{(g-1)p}^{2}(\alpha)
        =
        \chi_{(3-1)2}^{2}(0.05)
        =
        9.4877
    \]

    We have that $X_{\text{test (fac 1)}}^{2} = 19.8734 > X_{\text{crit (fac 1)}}^{2} = 9.4877$, so we would reject the null hypothesis that $\bm{\tau}_{1} = \bm{\tau}_{2} = \bm{\tau}_{3} = \textbf{0}$. It appears there is a factor 1 effect.

    \[
        \Lambda_{\text{fac 2}}^{\star}
        =
        \frac{\left| \text{SSP}_{\text{res}} \right|}{\left| \text{SSP}_{\text{fac 2}} + \text{SSP}_{\text{res}} \right|}
        =
        \frac{356}{6887}
        =
        0.0517
    \]

    \[
        X_{\text{test (fac 2)}}^{2}
        =
        -\left[(g-1)(b-1) - \frac{p+1-(b-1)}{2}\right]
        \ln \Lambda_{\text{fac 2}}^{\star}
        =
        17.7748
    \]
    
    \[
        X_{\text{crit (fac 2)}}^{2}
        =
        \chi_{(b-1)p}^{2}(\alpha)
        =
        \chi_{(4-1)2}^{2}(0.05)
        =
        12.5916
    \]

    We have that $X_{\text{test (fac 2)}}^{2} = 17.7748 > X_{\text{crit (fac 2)}}^{2} = 12.5916$, so we would reject the null hypothesis that $\bm{\beta}_{1} = \bm{\beta}_{2} = \bm{\beta}_{3} = \bm{\beta}_{4} = \textbf{0}$. It appears there is a factor 2 effect.
\end{enumerate}