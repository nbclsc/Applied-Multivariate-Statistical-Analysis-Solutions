A \textit{replicate} of the experiment in Exercise 6.13 yields the following data:

\begin{center}
    \begin{NiceTabular}{cccccc}[margin,cell-space-limits=1.5mm]
        & & \Block{1-4}{Factor 2} \\
        &
        &
        \Block[c]{}{Level \\ 1}
        &
        \Block[c]{}{Level \\ 2}
        &
        \Block[c]{}{Level \\ 3}
        &
        \Block[c]{}{Level \\ 4} \\
        \Block{3-1}{Factor 1} & Level 1 &
        $\left[
            \begin{array}{r}
                14 \\
                 8
            \end{array}
        \right]$
        &
        $\left[
            \begin{array}{r}
                6 \\
                2
            \end{array}
        \right]$
        &
        $\left[
            \begin{array}{r}
                8 \\
                2
            \end{array}
        \right]$
        &
        $\left[
            \begin{array}{r}
                16 \\
                -4
            \end{array}
        \right]$
        \\
        & Level 2 &
        $\left[
            \begin{array}{r}
                1 \\
                6
            \end{array}
        \right]$
        &
        $\left[
            \begin{array}{r}
                 5 \\
                12
            \end{array}
        \right]$
        &
        $
        \left[
            \begin{array}{r}
                 0 \\
                15
            \end{array}
        \right]$
        &
        $\left[
            \begin{array}{r}
                2 \\
                7
            \end{array}
        \right]$
        \\
        & Level 3 &
        $\left[
            \begin{array}{r}
                 3 \\
                -2
            \end{array}
        \right]$
        &
        $\left[
            \begin{array}{r}
                -2 \\
                 7
            \end{array}
        \right]$
        &
        $\left[
            \begin{array}{r}
                -11 \\
                  1
            \end{array}
        \right]$
        &
        $\left[
            \begin{array}{r}
                -6 \\
                 6
            \end{array}
        \right]$
        \CodeAfter \tikz \draw[solid] (1-|1) -- (1-|last);
                \tikz \draw[solid] (3-|1) -- (3-|last);
                \tikz \draw[solid] (last-|1) -- (last-|last);
                \tikz \draw[solid] (1-|3) -- (last-|3);
    \end{NiceTabular}
\end{center}
\begin{enumerate}[label= (\alph*)]
    \item Use these data to decompose each of the two measurements in the observation
    vector as
    \[
        x_{\ell k}
        =
        \bar{x}
        +
        (
            \bar{x}_{\ell \cdot}
            -
            \bar{x}
        )
        +
        (
            \bar{x}_{\cdot k}
            -
            \bar{x}
        )
        +
        (
            \bar{x}_{\ell k}
            -
            \bar{x}_{\ell \cdot}
            -
            \bar{x}_{\cdot k}
            +
            \bar{x}
        )
    \]
    where $\bar{x}$ is the overall average, $\bar{x}_{\ell \cdot}$ is the average for the $\ell$th
level of factor 1, and $\bar{x}_{\cdot k}$ is the average for the $k$th level of factor 2. Form the corresponding arrays for each of the two responses.
    \item Combine the preceding data with the data in Exercise 6.13 and carry out the necessary
    calculations to complete the general two-way MANOVA table.
    \item Given the results in Part b, test for interactions, and if the interactions do not
    exist, test for factor 1 and factor 2 main effects. Use the likelihood ratio test with
    $\alpha= .05$.
    \item If main effects, but no interactions, exist, examine the nature of the main effects by
    constructing Bonferroni simultaneous 95\% confidence intervals for differences of
    the components of the factor effect parameters.
\end{enumerate}