A \textit{replicate} of the experiment in Exercise 6.13 yields the following data:

\begin{center}
    \begin{NiceTabular}{cccccc}[margin,cell-space-limits=1.5mm]
        & & \Block{1-4}{Factor 2} \\
        &
        &
        \Block[c]{}{Level \\ 1}
        &
        \Block[c]{}{Level \\ 2}
        &
        \Block[c]{}{Level \\ 3}
        &
        \Block[c]{}{Level \\ 4} \\
        \Block{3-1}{Factor 1} & Level 1 &
        $\left[
            \begin{array}{r}
                14 \\
                 8
            \end{array}
        \right]$
        &
        $\left[
            \begin{array}{r}
                6 \\
                2
            \end{array}
        \right]$
        &
        $\left[
            \begin{array}{r}
                8 \\
                2
            \end{array}
        \right]$
        &
        $\left[
            \begin{array}{r}
                16 \\
                -4
            \end{array}
        \right]$
        \\
        & Level 2 &
        $\left[
            \begin{array}{r}
                1 \\
                6
            \end{array}
        \right]$
        &
        $\left[
            \begin{array}{r}
                 5 \\
                12
            \end{array}
        \right]$
        &
        $
        \left[
            \begin{array}{r}
                 0 \\
                15
            \end{array}
        \right]$
        &
        $\left[
            \begin{array}{r}
                2 \\
                7
            \end{array}
        \right]$
        \\
        & Level 3 &
        $\left[
            \begin{array}{r}
                 3 \\
                -2
            \end{array}
        \right]$
        &
        $\left[
            \begin{array}{r}
                -2 \\
                 7
            \end{array}
        \right]$
        &
        $\left[
            \begin{array}{r}
                -11 \\
                  1
            \end{array}
        \right]$
        &
        $\left[
            \begin{array}{r}
                -6 \\
                 6
            \end{array}
        \right]$
        \CodeAfter \tikz \draw[solid] (1-|1) -- (1-|last);
                \tikz \draw[solid] (3-|1) -- (3-|last);
                \tikz \draw[solid] (last-|1) -- (last-|last);
                \tikz \draw[solid] (1-|3) -- (last-|3);
    \end{NiceTabular}
\end{center}
\begin{enumerate}[label= (\alph*)]
    \item Use these data to decompose each of the two measurements in the observation
    vector as
    \[
        x_{\ell k}
        =
        \bar{x}
        +
        (
            \bar{x}_{\ell \cdot}
            -
            \bar{x}
        )
        +
        (
            \bar{x}_{\cdot k}
            -
            \bar{x}
        )
        +
        (
            \bar{x}_{\ell k}
            -
            \bar{x}_{\ell \cdot}
            -
            \bar{x}_{\cdot k}
            +
            \bar{x}
        )
    \]
    where $\bar{x}$ is the overall average, $\bar{x}_{\ell \cdot}$ is the average for the $\ell$th
    level of factor 1, and $\bar{x}_{\cdot k}$ is the average for the $k$th level of factor 2. Form the corresponding arrays for each of the two responses.
    \[
        \bar{\textbf{x}}
        =
        \bar{\textbf{x}}_{\cdot\cdot}
        =
        \begin{bNiceArray}{c}
            2 \\
            4
        \end{bNiceArray}
    \]
    
    \[
        \bar{\textbf{X}}_{\text{Factor 1}}
        =
        \begin{bNiceArray}{c}
            \bar{\textbf{x}}_{1 \cdot}^{\prime} \\
            \bar{\textbf{x}}_{2 \cdot}^{\prime} \\
            \bar{\textbf{x}}_{3 \cdot}^{\prime}
        \end{bNiceArray}
        =
        \begin{bNiceArray}{rr}[margin]
             8 & 5 \\
             1 & 7 \\
            -3 & 0
        \end{bNiceArray}
        \hspace{0.2cm}
        \text{and}
        \hspace{0.2cm}
        \bar{\textbf{X}}_{\text{Factor 2}}
        =
        \begin{bNiceArray}{c}
            \bar{\textbf{x}}_{\cdot 1}^{\prime} \\
            \bar{\textbf{x}}_{\cdot 2}^{\prime} \\
            \bar{\textbf{x}}_{\cdot 3}^{\prime} \\
            \bar{\textbf{x}}_{\cdot 4}^{\prime}
        \end{bNiceArray}
        =
        \begin{bNiceArray}{rr}[margin]
            4 & 5 \\
            1 & 4 \\
            2 & 5 \\
            1 & 2
        \end{bNiceArray}
    \]
    \[
        \bar{\textbf{X}}_{\ell k}
        =
        \begin{bNiceArray}{ccc}
            \bar{\textbf{x}}_{11} & \bar{\textbf{x}}_{12} & \bar{\textbf{x}}_{13} \\
            \bar{\textbf{x}}_{21} & \bar{\textbf{x}}_{22} & \bar{\textbf{x}}_{23} \\
            \bar{\textbf{x}}_{31} & \bar{\textbf{x}}_{32} & \bar{\textbf{x}}_{33}
        \end{bNiceArray}
        =
        \begin{bNiceMatrix}[margin]
            10 &  5 &  8 &  9 \\
             8 &  4 &  7 &  1 \\
             \\
             2 &  1 &  2 & -1 \\
             7 &  7 &  9 &  5 \\
             \\
             0 & -3 & -4 & -5 \\
             0 &  1 & -1 &  0
            \CodeAfter
             \SubMatrix[{1-1}{2-1}]
             \SubMatrix[{1-2}{2-2}]
             \SubMatrix[{1-3}{2-3}]
             \SubMatrix[{1-4}{2-4}]
             \SubMatrix[{4-1}{5-1}]
             \SubMatrix[{4-2}{5-2}]
             \SubMatrix[{4-3}{5-3}]
             \SubMatrix[{4-4}{5-4}]
             \SubMatrix[{7-1}{8-1}]
             \SubMatrix[{7-2}{8-2}]
             \SubMatrix[{7-3}{8-3}]
             \SubMatrix[{7-4}{8-4}]
        \end{bNiceMatrix}
    \]
       
    \underline{Observation 1, variable 1:}
    \begin{multline*}
        \underset{\text{(observation)}}{
            \left[
                \begin{array}{rrrr}
                    6 &  4 & 8 &  2 \\
                    3 & -3 & 4 & -4 \\
                    -3 & -4 & 3 & -4
                \end{array}
            \right]
        }
        =
        \\
        \underset{\text{(mean)}}{
            \left[
                \begin{array}{rrrr}
                    2 & 2 & 2 & 2 \\
                    2 & 2 & 2 & 2 \\
                    2 & 2 & 2 & 2
                \end{array}
            \right]
        }
        +
        \underset{\text{(treatment 1 effect)}}{
            \left[
                \begin{array}{rrrr}
                     6 &  6 &  6 &  6 \\
                    -1 & -1 & -1 & -1 \\
                    -5 & -5 & -5 & -5
                \end{array}
            \right]
        }
        \\
        +
        \underset{\text{(treatment 2 effect)}}{
            \left[
                \begin{array}{rrrr}
                    2 & -1 & 0 & -1 \\
                    2 & -1 & 0 & -1 \\
                    2 & -1 & 0 & -1
                \end{array}
            \right]
        }
        +
        \underset{\text{(interaction)}}{
            \left[
                \begin{array}{rrrr}
                     0 & -2 &  0 &  2 \\
                    -1 &  1 &  1 & -1 \\
                     1 &  1 & -1 & -1
                \end{array}
            \right]
        }
        \\
        +
        \underset{\text{(residual)}}{
            \left[
                \begin{array}{rrrr}
                    -4 & -1 &  0 & -7 \\
                     1 & -4 &  2 & -3 \\
                    -3 & -1 &  7 &  1
                \end{array}
            \right]
        }
    \end{multline*}

    \underline{Observation 1, variable 2:}
    \begin{multline*}
        \underset{\text{(observation)}}{
            \left[
                \begin{array}{rrrr}
                    8 &  6 & 12 &  6 \\
                    8 &  2 &  3 &  3 \\
                    2 & -5 & -3 & -6
                \end{array}
            \right]
        }
        \\
        =
        \underset{\text{(mean)}}{
            \left[
                \begin{array}{rrrr}
                    4 & 4 & 4 & 4 \\
                    4 & 4 & 4 & 4 \\
                    4 & 4 & 4 & 4
                \end{array}
            \right]
        }
        +
        \underset{\text{(treatment 1 effect)}}{
            \left[
                \begin{array}{rrrr}
                     1 &  1 &  1 &  1 \\
                     3 &  3 &  3 &  3 \\
                    -4 & -4 & -4 & -4
                \end{array}
            \right]
        }
        \\
        +
        \underset{\text{(treatment 2 effect)}}{
            \left[
                \begin{array}{rrrr}
                    1 &  0 & 1 & -2 \\
                    1 &  0 & 1 & -2 \\
                    1 &  0 & 1 & -2
                \end{array}
            \right]
        }
        +
        \underset{\text{(interaction)}}{
            \left[
                \begin{array}{rrrr}
                     2 & -1 &  1 & -2 \\
                    -1 &  0 &  1 &  0 \\
                    -1 &  1 & -2 &  2
                \end{array}
            \right]
        }
        \\
        +
        \underset{\text{(residual)}}{
            \left[
                \begin{array}{rrrr}
                     0 &  2 &  5 &  5 \\
                     1 & -5 & -6 & -2 \\
                     2 & -6 & -2 & -6
                \end{array}
            \right]
        }
    \end{multline*}

    \underline{Observation 2, variable 1:}
    \begin{multline*}
        \underset{\text{(observation)}}{
            \left[
                \begin{array}{rrrr}
                    14 &  6 &   8 & 16 \\
                     1 &  5 &   0 &  2 \\
                     3 & -2 & -11 & -6
                \end{array}
            \right]
        }
        \\
        =
        \underset{\text{(mean)}}{
            \left[
                \begin{array}{rrrr}
                    2 & 2 & 2 & 2 \\
                    2 & 2 & 2 & 2 \\
                    2 & 2 & 2 & 2
                \end{array}
            \right]
        }
        +
        \underset{\text{(treatment 1 effect)}}{
            \left[
                \begin{array}{rrrr}
                     6 &  6 &  6 &  6 \\
                    -1 & -1 & -1 & -1 \\
                    -5 & -5 & -5 & -5 
                \end{array}
            \right]
        }
        \\
        +
        \underset{\text{(treatment 2 effect)}}{
            \left[
                \begin{array}{rrrr}
                    2 & -1 & 0 & -1 \\
                    2 & -1 & 0 & -1 \\
                    2 & -1 & 0 & -1
                \end{array}
            \right]
        }
        +
        \underset{\text{(interaction)}}{
            \left[
                \begin{array}{rrrr}
                     0 & -2 &  0 &  2 \\
                    -1 &  1 &  1 & -1 \\
                     1 &  1 & -1 & -1
                \end{array}
            \right]
        }
        \\
        +
        \underset{\text{(residual)}}{
            \left[
                \begin{array}{rrrr}
                     4 &  1 &  0 &  7 \\
                    -1 &  4 & -2 &  3 \\
                     3 &  1 & -7 & -1
                \end{array}
            \right]
        }
    \end{multline*}
    
    \underline{Observation 2, variable 2:}
    \begin{multline*}
        \underset{\text{(observation)}}{
            \left[
                \begin{array}{rrrr}
                     8 &  2 &   2 & -4 \\
                     6 & 12 &  15 &  7 \\
                    -2 &  7 &   1 &  6
                \end{array}
            \right]
        }
        \\
        =
        \underset{\text{(mean)}}{
            \left[
                \begin{array}{rrrr}
                    4 & 4 & 4 & 4 \\
                    4 & 4 & 4 & 4 \\
                    4 & 4 & 4 & 4
                \end{array}
            \right]
        }
        +
        \underset{\text{(treatment 1 effect)}}{
            \left[
                \begin{array}{rrrr}
                     1 &  1 &  1 &  1 \\
                     3 &  3 &  3 &  3 \\
                    -4 & -4 & -4 & -4 
                \end{array}
            \right]
        }
        \\
        +
        \underset{\text{(treatment 2 effect)}}{
            \left[
                \begin{array}{rrrr}
                    1 & 0 & 1 & -2 \\
                    1 & 0 & 1 & -2 \\
                    1 & 0 & 1 & -2
                \end{array}
            \right]
        }
        +
        \underset{\text{(interaction)}}{
            \left[
                \begin{array}{rrrr}
                     2 & -1 &  1 & -2 \\
                    -1 &  0 &  1 &  0 \\
                    -1 &  1 & -2 &  2
                \end{array}
            \right]
        }
        \\
        +
        \underset{\text{(residual)}}{
            \left[
                \begin{array}{rrrr}
                     0 & -2 & -5 & -5 \\
                    -1 &  5 &  6 &  2 \\
                    -2 &  6 &  2 &  6
                \end{array}
            \right]
        }
    \end{multline*}

    \item Combine the preceding data with the data in Exercise 6.13 and carry out the necessary
    calculations to complete the general two-way MANOVA table.

    What they're saying here is to apply the Vec operator to the matrices and take the dot product to complete the computations, but instead, I'm just going use the formula summarized in the solution to Exercise 6.8 to do this.
    That is applying the formula below to our matrices
    \[
        \left[
            \begin{array}{cc}
                \text{sum}(\textbf{A}_{1} \circ \textbf{A}_{1}) & \text{sum}(\textbf{A}_{1} \circ \textbf{A}_{2}) \\
                \text{sum}(\textbf{A}_{2} \circ \textbf{A}_{1}) & \text{sum}(\textbf{A}_{2} \circ \textbf{A}_{2})
            \end{array}
        \right]
        =
        \left[
            \begin{array}{cc}
                \text{tr}(\textbf{A}_{1} \textbf{A}_{1}^{\prime}) & \text{tr}(\textbf{A}_{1} \textbf{A}_{2}^{\prime}) \\
                \text{tr}(\textbf{A}_{2} \textbf{A}_{1}^{\prime}) & \text{tr}(\textbf{A}_{2} \textbf{A}_{2}^{\prime})
            \end{array}
        \right]
    \]

    \begin{align*}
        \text{tr}(\textbf{T}_{11} \textbf{T}_{11}^{\prime})
        & =
        \text{tr}
        \left(
            \begin{bNiceArray}{rrrr}
                 6 &  6 &  6 &  6 \\
                -1 & -1 & -1 & -1 \\
                -5 & -5 & -5 & -5
            \end{bNiceArray}
            \begin{bNiceArray}{rrr}
                6 & -1 & -5 \\
                6 & -1 & -5 \\
                6 & -1 & -5 \\
                6 & -1 & -5
            \end{bNiceArray}
    \right) \\
    & =
    \text{tr}
        \left(
            \begin{bNiceArray}{rrr}
                 144 & -24 & -120 \\
                -24  &   4 &   20 \\
                -120 &  20 &  100
            \end{bNiceArray}
        \right) \\
        & =
        248
    \end{align*}

    \begin{align*}
        \text{tr}(\textbf{T}_{11} \textbf{T}_{21}^{\prime})
        & =
        \text{tr}(\textbf{T}_{21} \textbf{T}_{11}^{\prime}) \\
        & =
        \text{tr}
        \left(
            \begin{bNiceArray}{rrrr}
                 6 &  6 &  6 &  6 \\
                -1 & -1 & -1 & -1 \\
                -5 & -5 & -5 & -5
            \end{bNiceArray}
           \begin{bNiceArray}{rrr}
               1 & 3 & -4 \\
               1 & 3 & -4 \\
               1 & 3 & -4 \\
               1 & 3 & -4
           \end{bNiceArray}
    \right) \\
    & =
    \text{tr}
        \left(
            \begin{bNiceArray}{rrr}
                 24 &  72 & -96 \\
                 -4 & -12 &  16 \\
                -20 & -60 &  80
            \end{bNiceArray}
        \right) \\
        & =
        92
    \end{align*}

    \begin{align*}
        \text{tr}(\textbf{T}_{21} \textbf{T}_{21}^{\prime})
        & =
        \text{tr}
        \left(
            \begin{bNiceArray}{rrrr}
                1 &  1 &  1 &  1 \\
                3 &  3 &  3 &  3 \\
               -4 & -4 & -4 & -4
           \end{bNiceArray}
            \begin{bNiceArray}{rrr}
               1 & 3 & -4 \\
               1 & 3 & -4 \\
               1 & 3 & -4 \\
               1 & 3 & -4
           \end{bNiceArray}
    \right) \\
    & =
    \text{tr}
        \left(
            \begin{bNiceArray}{rrr}
                  4 &  12 & -16 \\
                 12 &  36 & -48 \\
                -16 & -48 &  64
            \end{bNiceArray}
        \right) \\
        & =
        104
    \end{align*}

    \begin{align*}
        \textbf{SSP}_{\text{fac 1}}
        & =
        \textbf{B}_{1}
        =
        \begin{bNiceArray}{cc}
            \text{tr}(\textbf{T}_{11} \textbf{T}_{11}^{\prime}) & \text{tr}(\textbf{T}_{11} \textbf{T}_{21}^{\prime}) \\
            \text{tr}(\textbf{T}_{21} \textbf{T}_{11}^{\prime}) & \text{tr}(\textbf{T}_{21} \textbf{T}_{21}^{\prime})
        \end{bNiceArray}
        \\
        & =
        \begin{bNiceArray}{cc}
            n*248 &  n*92 \\
             n*92 & n*104
        \end{bNiceArray}
        =
        \begin{bNiceArray}{cc}
            2*248 &  2*92 \\
             2*92 & 2*104
        \end{bNiceArray}
        \\
        & =
        \begin{bNiceArray}{cc}
            496 & 184 \\
            184 & 208
        \end{bNiceArray}
    \end{align*}

    \begin{align*}
        \text{tr}(\textbf{T}_{12} \textbf{T}_{12}^{\prime})
        & =
        \text{tr}
        \left(
            \begin{bNiceArray}{rrrr}
                2 & -1 & 0 & -1 \\
                2 & -1 & 0 & -1 \\
                2 & -1 & 0 & -1
            \end{bNiceArray}
            \begin{bNiceArray}{rrr}
                 2 &  2 &  2 \\
                -1 & -1 & -1 \\
                 0 &  0 &  0 \\
                -1 & -1 & -1
            \end{bNiceArray}
    \right) \\
    & =
    \text{tr}
        \left(
            \begin{bNiceArray}{rrr}
                6 & 6 & 6 \\
                6 & 6 & 6 \\
                6 & 6 & 6
            \end{bNiceArray}
        \right) \\
        & =
        18
    \end{align*}

    \begin{align*}
        \text{tr}(\textbf{T}_{12} \textbf{T}_{22}^{\prime})
        & =
        \text{tr}(\textbf{T}_{22} \textbf{T}_{12}^{\prime}) \\
        & =
        \text{tr}
        \left(
            \begin{bNiceArray}{rrrr}
                2 & -1 & 0 & -1 \\
                2 & -1 & 0 & -1 \\
                2 & -1 & 0 & -1
            \end{bNiceArray}
            \begin{bNiceArray}{rrr}
                 1 &  1 &  1 \\
                 0 &  0 &  0 \\
                 1 &  1 &  1 \\
                -2 & -2 & -2
            \end{bNiceArray}
    \right) \\
    & =
    \text{tr}
        \left(
            \begin{bNiceArray}{rrr}
                4 & 4 & 4 \\
                4 & 4 & 4 \\
                4 & 4 & 4
            \end{bNiceArray}
        \right) \\
        & =
        12
    \end{align*}

    \begin{align*}
        \text{tr}(\textbf{T}_{22} \textbf{T}_{22}^{\prime})
        & =
        \text{tr}
        \left(
            \begin{bNiceArray}{rrrr}
                1 & 0 & 1 & -2 \\
                1 & 0 & 1 & -2 \\
                1 & 0 & 1 & -2
            \end{bNiceArray}
             \begin{bNiceArray}{rrr}
                 1 &  1 &  1 \\
                 0 &  0 &  0 \\
                 1 &  1 &  1 \\
                -2 & -2 & -2
            \end{bNiceArray}
    \right) \\
    & =
    \text{tr}
        \left(
            \begin{bNiceArray}{rrr}
                6 & 6 & 6 \\
                6 & 6 & 6 \\
                6 & 6 & 6
            \end{bNiceArray}
        \right) \\
        & =
        18
    \end{align*}

    \begin{align*}
        \textbf{SSP}_{\text{fac 2}}
        & =
        \textbf{B}_{2}
        =
        \begin{bNiceArray}{cc}
            \text{tr}(\textbf{T}_{12} \textbf{T}_{12}^{\prime}) & \text{tr}(\textbf{T}_{12} \textbf{T}_{22}^{\prime}) \\
            \text{tr}(\textbf{T}_{22} \textbf{T}_{12}^{\prime}) & \text{tr}(\textbf{T}_{22} \textbf{T}_{22}^{\prime})
        \end{bNiceArray}
        \\
        & =
        \begin{bNiceArray}{cc}
            n*18 & n*12 \\
            n*12 & n*18
        \end{bNiceArray}
        =
        \begin{bNiceArray}{cc}
            2*18 & 2*12 \\
            2*12 & 2*18
        \end{bNiceArray}
        \\
        & =
        \begin{bNiceArray}{cc}
            36 & 24 \\
            24 & 36
        \end{bNiceArray}
    \end{align*}

    \begin{align*}
        \text{tr}(\textbf{I}_{1} \textbf{I}_{1}^{\prime})
        & =
        \text{tr}
        \left(
            \begin{bNiceArray}{rrrr}
                 0 & -2 &  0 &  2 \\
                -1 &  1 &  1 & -1 \\
                 1 &  1 & -1 & -1
            \end{bNiceArray}
            \begin{bNiceArray}{rrr}
                 0 & -1 &  1 \\
                -2 &  1 &  1 \\
                 0 &  1 & -1 \\
                 2 & -1 & -1
            \end{bNiceArray}
    \right) \\
    & =
    \text{tr}
        \left(
            \begin{bNiceArray}{rrr}
                 8 & -4 & -4 \\
                -4 &  4 &  0 \\
                -4 &  0 &  4
            \end{bNiceArray}
        \right) \\
        & =
        16
    \end{align*}

    \begin{align*}
        \text{tr}(\textbf{I}_{1} \textbf{I}_{2}^{\prime})
        & =
        \text{tr}(\textbf{I}_{2} \textbf{I}_{1}^{\prime}) \\
        & =
        \text{tr}
        \left(
             \begin{bNiceArray}{rrrr}
                 0 & -2 &  0 &  2 \\
                -1 &  1 &  1 & -1 \\
                 1 &  1 & -1 & -1
            \end{bNiceArray}
            \begin{bNiceArray}{rrr}
                 2 & -1 & -1 \\
                -1 &  0 &  1 \\
                 1 &  1 & -2 \\
                -2 &  0 &  2
            \end{bNiceArray}
    \right) \\
    & =
    \text{tr}
        \left(
            \begin{bNiceArray}{rrr}
                -2 &  0 &  2 \\
                 0 &  2 & -2 \\
                 2 & -2 &  0
            \end{bNiceArray}
        \right) \\
        & =
        0
    \end{align*}

    \begin{align*}
        \text{tr}(\textbf{I}_{2} \textbf{I}_{2}^{\prime})
        & =
        \text{tr}
        \left(
            \begin{bNiceArray}{rrrr}
                 2 & -1 &  1 & -2 \\
                -1 &  0 &  1 &  0 \\
                -1 &  1 & -2 &  2
            \end{bNiceArray}
             \begin{bNiceArray}{rrr}
                 2 & -1 & -1 \\
                -1 &  0 &  1 \\
                 1 &  1 & -2 \\
                -2 &  0 &  2
            \end{bNiceArray}
    \right) \\
    & =
    \text{tr}
        \left(
            \begin{bNiceArray}{rrr}
                10 & -1 & -9 \\
                -1 &  2 & -1 \\
                -9 & -1 & 10
            \end{bNiceArray}
        \right) \\
        & =
        22
    \end{align*}

    \begin{align*}
        \textbf{SSP}_{\text{interaction}}
        & =
        \textbf{I}
        =
        \begin{bNiceArray}{cc}
            \text{tr}(\textbf{I}_{1} \textbf{I}_{1}^{\prime}) & \text{tr}(\textbf{I}_{1} \textbf{I}_{2}^{\prime}) \\
            \text{tr}(\textbf{I}_{2} \textbf{I}_{1}^{\prime}) & \text{tr}(\textbf{I}_{2} \textbf{I}_{2}^{\prime})
        \end{bNiceArray}
        \\
        & =
        \begin{bNiceArray}{cc}
            n*16 & n*0 \\
            n*0 & n*22
        \end{bNiceArray}
        =
        \begin{bNiceArray}{cc}
            2*16 & 2*0 \\
            2*0 & 2*22
        \end{bNiceArray}
        \\
        & =
        \begin{bNiceArray}{cc}
            32 &  0 \\
             0 & 44
        \end{bNiceArray}
    \end{align*}

  For the residuals, we can see above that the residuals for observation two are negative of the residuals for observation one. Taking the trace of the product will be the same no matter which observation we chose to compute the within SS and cross-products for, so I'm using observation one.

    \begin{align*}
        \text{tr}(\textbf{E}_{1} \textbf{E}_{1}^{\prime})
        & =
        \text{tr}
        \left(
            \begin{bNiceArray}{rrrr}
                -4 & -1 &  0 & -7 \\
                 1 & -4 &  2 & -3 \\
                -3 & -1 &  7 &  1
            \end{bNiceArray}
            \begin{bNiceArray}{rrr}
                -4 &  1 & -3 \\
                -1 & -4 & -1 \\
                 0 &  2 &  7 \\
                -7 & -3 &  1
            \end{bNiceArray}
    \right) \\
    & =
    \text{tr}
        \left(
            \begin{bNiceArray}{rrr}
                66 & 21 &  6 \\
                21 & 30 & 12 \\
                 6 & 12 & 60
            \end{bNiceArray}
        \right) \\
        & =
        156
    \end{align*}

    \begin{align*}
        \text{tr}(\textbf{E}_{1} \textbf{E}_{2}^{\prime})
        & =
        \text{tr}(\textbf{E}_{2} \textbf{E}_{1}^{\prime}) \\
        & =
        \text{tr}
        \left(
             \begin{bNiceArray}{rrrr}
                -4 & -1 &  0 & -7 \\
                 1 & -4 &  2 & -3 \\
                -3 & -1 &  7 &  1
            \end{bNiceArray}
            \begin{bNiceArray}{rrr}
                0 &  1 &  2 \\
                2 & -5 & -6 \\
                5 & -6 & -2 \\
                5 & -2 & -6
           \end{bNiceArray}
    \right) \\
    & =
    \text{tr}
        \left(
            \begin{bNiceArray}{rrr}
                -37 &  15 &  40 \\
                -13 &  15 &  40 \\
                 38 & -42 & -20
            \end{bNiceArray}
        \right) \\
        & =
        -42
    \end{align*}

    \begin{align*}
        \text{tr}(\textbf{E}_{2} \textbf{E}_{2}^{\prime})
        & =
        \text{tr}
        \left(
            \begin{bNiceArray}{rrrr}
                0 &  2 &  5 &  5 \\
                1 & -5 & -6 & -2 \\
                2 & -6 & -2 & -6
           \end{bNiceArray}
             \begin{bNiceArray}{rrr}
                0 &  1 &  2 \\
                2 & -5 & -6 \\
                5 & -6 & -2 \\
                5 & -2 & -6
           \end{bNiceArray}
     \right)
     \\
    & =
    \text{tr}
        \left(
            \begin{bNiceArray}{rrr}
                 54 & -50 & -52 \\
                -50 &  66 &  56 \\
                -52 &  56 &  80
            \end{bNiceArray}
        \right) \\
        & =
        200
    \end{align*}

    \begin{align*}
        \textbf{SSP}_{\text{res}}
        &
        =
        \textbf{W}
        =
        \begin{bNiceArray}{cc}
            \text{tr}(\textbf{E}_{1} \textbf{E}_{1}^{\prime}) & \text{tr}(\textbf{E}_{1} \textbf{E}_{2}^{\prime}) \\
            \text{tr}(\textbf{E}_{2} \textbf{E}_{1}^{\prime}) & \text{tr}(\textbf{E}_{2} \textbf{E}_{2}^{\prime})
        \end{bNiceArray}
        \\
        & =
        \begin{bNiceArray}{cc}
            n*156 & n*(-42) \\
            n*(-42) & n*200
        \end{bNiceArray}
        =
        \begin{bNiceArray}{cc}
            2*156 & 2*(-42) \\
            2*(-42) & 2*200
        \end{bNiceArray}
        \\
        & =
        \begin{bNiceArray}{cc}
            312 & -84 \\
            -84 & 400
        \end{bNiceArray}
    \end{align*}


To compute the $\textbf{SSP}_{\text{cor}}$, we need the observations and mean sum of squares and sum of cross-product matrices.
The observation matrix is computed as

    \underline{Observation 1:}
    \begin{align*}
        \text{tr}(\textbf{O}_{11} \textbf{O}_{11}^{\prime})
        & =
        \text{tr}
        \left(
            \begin{bNiceArray}{rrrr}
                 6 &  4 & 8 &  2 \\
                 3 & -3 & 4 & -4 \\
                -3 & -4 & 3 & -4
            \end{bNiceArray}
            \begin{bNiceArray}{rrr}
                6 &  3 & -3 \\
                4 & -3 & -4 \\
                8 &  4 &  3 \\
                2 & -4 &  4
            \end{bNiceArray}
    \right) \\
    & =
    \text{tr}
        \left(
            \begin{bNiceArray}{rrr}
                120 & 30 & -18 \\
                 30 & 50 &  31 \\
                -18 & 31 &  50
            \end{bNiceArray}
        \right) \\
        & =
        220
    \end{align*}

    \begin{align*}
        \text{tr}(\textbf{O}_{11} \textbf{O}_{12}^{\prime})
        & =
        \text{tr}(\textbf{O}_{12} \textbf{O}_{11}^{\prime}) \\
        & =
        \text{tr}
        \left(
            \begin{bNiceArray}{rrrr}
                6 &  4 & 8 &  2 \\
                3 & -3 & 4 & -4 \\
               -3 & -4 & 3 & -4
           \end{bNiceArray}
           \begin{bNiceArray}{rrr}
                8 & 8 &  2 \\
                6 & 2 & -5 \\
               12 & 3 & -3 \\
                6 & 3 & -6
           \end{bNiceArray}
    \right) \\
    & =
    \text{tr}
        \left(
            \begin{bNiceArray}{rrr}
                180 &  86 & -44 \\
                 30 &  18 &  33 \\
                -36 & -35 &  29
            \end{bNiceArray}
        \right) \\
        & =
        227
    \end{align*}

    \begin{align*}
        \text{tr}(\textbf{O}_{12} \textbf{O}_{12}^{\prime})
        & =
        \text{tr}
        \left(
            \begin{bNiceArray}{rrrr}
                8 &  6 & 12 &  6 \\
                8 &  2 &  3 &  3 \\
                2 & -5 & -3 & -6
           \end{bNiceArray}
           \begin{bNiceArray}{rrr}
                8 & 8 &  2 \\
                6 & 2 & -5 \\
               12 & 3 & -3 \\
                6 & 3 & -6
           \end{bNiceArray}
    \right) \\
    & =
    \text{tr}
        \left(
            \begin{bNiceArray}{rrr}
                280 & 130 & -86 \\
                130 &  86 & -21 \\
                -86 & -21 &  74
            \end{bNiceArray}
        \right) \\
        & =
        440
    \end{align*}

    \underline{Observation 2:}
    \begin{align*}
        \text{tr}(\textbf{O}_{21} \textbf{O}_{21}^{\prime})
        & =
        \text{tr}
        \left(
            \begin{bNiceArray}{rrrr}
                14 &  6 &   8 & 16 \\
                 1 &  5 &   0 &  2 \\
                 3 & -2 & -11 & -6
            \end{bNiceArray}
            \begin{bNiceArray}{rrr}
                14 &  1 &   3 \\
                 6 &  5 &  -2 \\
                 8 &  0 & -11 \\
                16 &  2 &  -6
            \end{bNiceArray}
    \right) \\
    & =
    \text{tr}
        \left(
            \begin{bNiceArray}{rrr}
                 552 &  76 & -154 \\
                  76 &  30 &  -19 \\
                -154 & -19 &  170
            \end{bNiceArray}
        \right) \\
        & =
        752
    \end{align*}

    \begin{align*}
        \text{tr}(\textbf{O}_{21} \textbf{O}_{22}^{\prime})
        & =
        \text{tr}(\textbf{O}_{22} \textbf{O}_{21}^{\prime}) \\
        & =
        \text{tr}
        \left(
            \begin{bNiceArray}{rrrr}
                14 &  6 &   8 & 16 \\
                 1 &  5 &   0 &  2 \\
                 3 & -2 & -11 & -6
            \end{bNiceArray}
           \begin{bNiceArray}{rrr}
                 8 &  6 & -2 \\
                 2 & 12 &  7 \\
                 2 & 15 &  1 \\
                -4 &  7 &  6
           \end{bNiceArray}
    \right) \\
    & =
    \text{tr}
        \left(
            \begin{bNiceArray}{rrr}
                76 &  388 & 118 \\
                10 &   80 &  45 \\
                22 & -213 & -67
            \end{bNiceArray}
        \right) \\
        & =
        89
    \end{align*}

    \begin{align*}
        \text{tr}(\textbf{O}_{22} \textbf{O}_{22}^{\prime})
        & =
        \text{tr}
        \left(
            \begin{bNiceArray}{rrrr}
                 8 &  2 &  2 & -4 \\
                 6 & 12 & 15 &  7 \\
                -2 &  7 &  1 &  6
           \end{bNiceArray}
           \begin{bNiceArray}{rrr}
                 8 &  6 & -2 \\
                 2 & 12 &  7 \\
                 2 & 15 &  1 \\
                -4 &  7 &  6
           \end{bNiceArray}
    \right) \\
    & =
    \text{tr}
        \left(
            \begin{bNiceArray}{rrr}
                 88 &  74 & -24 \\
                 74 & 454 & 129 \\
                -24 & 129 &  90
            \end{bNiceArray}
        \right) \\
        & =
        632
    \end{align*}

    \begin{align*}
        \textbf{SSP}_{\text{obs}}
        & =
        \begin{bNiceArray}{cc}
            \sum_{r=1}^{2}\text{tr}(\textbf{O}_{r1} \textbf{O}_{r1}^{\prime}) & \sum_{r=1}^{2}\text{tr}(\textbf{O}_{r1} \textbf{O}_{r2}^{\prime}) \\
            \sum_{r=1}^{2}\text{tr}(\textbf{O}_{r2} \textbf{O}_{r1}^{\prime}) & \sum_{r=1}^{2}\text{tr}(\textbf{O}_{r2} \textbf{O}_{r2}^{\prime})
        \end{bNiceArray}
        \\
        & =
        \begin{bNiceArray}{rr}
            220+752 & 227+89 \\
            227+89 & 440+632
        \end{bNiceArray}
        =
        \begin{bNiceArray}{rr}
            972 &  316 \\
            316 & 1072
        \end{bNiceArray}
    \end{align*}

    The mean matrix is computed as
    \begin{align*}
        \text{tr}(\textbf{M}_{1} \textbf{M}_{1}^{\prime})
        & =
        \text{tr}
        \left(
            \begin{bNiceArray}{rrrr}
                2 & 2 & 2 & 2 \\
                2 & 2 & 2 & 2 \\
                2 & 2 & 2 & 2
            \end{bNiceArray}
            \begin{bNiceArray}{rrr}
                2 & 2 & 2 \\
                2 & 2 & 2 \\
                2 & 2 & 2 \\
                2 & 2 & 2
            \end{bNiceArray}
    \right) \\
    & =
    \text{tr}
        \left(
            \begin{bNiceArray}{rrr}
                16 & 16 & 16 \\
                16 & 16 & 16 \\
                16 & 16 & 16
            \end{bNiceArray}
        \right) \\
        & =
        48
    \end{align*}

    \begin{align*}
        \text{tr}(\textbf{M}_{1} \textbf{M}_{2}^{\prime})
        & =
        \text{tr}(\textbf{M}_{2} \textbf{M}_{1}^{\prime}) \\
        & =
        \text{tr}
        \left(
            \begin{bNiceArray}{rrrr}
                2 & 2 & 2 & 2 \\
                2 & 2 & 2 & 2 \\
                2 & 2 & 2 & 2
            \end{bNiceArray}
            \begin{bNiceArray}{rrr}
                4 & 4 & 4 \\
                4 & 4 & 4 \\
                4 & 4 & 4 \\
                4 & 4 & 4
            \end{bNiceArray}
    \right) \\
    & =
    \text{tr}
        \left(
            \begin{bNiceArray}{rrr}
                32 & 32 & 32 \\
                32 & 32 & 32 \\
                32 & 32 & 32
            \end{bNiceArray}
        \right) \\
        & =
        96
    \end{align*}

    \begin{align*}
        \text{tr}(\textbf{M}_{2} \textbf{M}_{2}^{\prime})
        & =
        \text{tr}
        \left(
            \begin{bNiceArray}{rrrr}
                4 & 4 & 4 & 4 \\
                4 & 4 & 4 & 4 \\
                4 & 4 & 4 & 4
            \end{bNiceArray}
            \begin{bNiceArray}{rrr}
                4 & 4 & 4 \\
                4 & 4 & 4 \\
                4 & 4 & 4 \\
                4 & 4 & 4
            \end{bNiceArray}
    \right) \\
    & =
    \text{tr}
        \left(
            \begin{bNiceArray}{rrr}
                64 & 64 & 64 \\
                64 & 64 & 64 \\
                64 & 64 & 64
            \end{bNiceArray}
        \right) \\
        & =
        192
    \end{align*}

    \begin{align*}
        \textbf{SSP}_{\text{mean}}
        & =
        \begin{bNiceArray}{cc}
            \text{tr}(\textbf{M}_{1} \textbf{M}_{1}^{\prime}) & \text{tr}(\textbf{M}_{1} \textbf{M}_{2}^{\prime}) \\
            \text{tr}(\textbf{M}_{2} \textbf{M}_{1}^{\prime}) & \text{tr}(\textbf{M}_{2} \textbf{M}_{2}^{\prime})
        \end{bNiceArray}
        \\
        & =
        \begin{bNiceArray}{rr}
            n \times 48 & n \times 96  \\
            n \times 96 & n \times 192
        \end{bNiceArray}
        =
        \begin{bNiceArray}{rr}
            2 \times 48 & 2 \times 96  \\
            2 \times 96 & 2 \times 192
        \end{bNiceArray}
        \\
        & =
        \begin{bNiceArray}{rr}
             96 & 192  \\
            192 & 384
        \end{bNiceArray}
    \end{align*}

    \begin{align*}
        \textbf{SSP}_{\text{cor}}
        & =
        \textbf{SSP}_{\text{obs}}
        -
        \textbf{SSP}_{\text{mean}}
        \\
        & =
        \begin{bNiceArray}{rr}
            972 &  316 \\
            316 & 1072
        \end{bNiceArray}
        -
        \begin{bNiceArray}{rr}
             96 & 192  \\
            192 & 384
        \end{bNiceArray}
        \\
        & =
        \begin{bNiceArray}{rr}
            876 & 124 \\
            124 & 688
        \end{bNiceArray}
    \end{align*}
    Finally putting all this together for the MANOVA table
    \[
    \begin{array}{lll}
        \text{Source} & \text{Matrix of sum of squares} & \\
        \text{of variation} & \text{and cross products} & \text{Degrees of freedom} \\
        \hline \\
        \text{Factor 1}
        &
        \begin{bNiceArray}{rr}
            496 & 184 \\
            184 & 208
        \end{bNiceArray}
        &
        3 - 1 = 2
        \\
        \text{Factor 2}
        &
        \begin{bNiceArray}{rr}
            36 & 24 \\
            24 & 36
        \end{bNiceArray}
        &
         4 - 1 = 3
        \\
        \text{Interaction}
        &
        \begin{bNiceArray}{rr}
            32 &  0 \\
             0 & 44
        \end{bNiceArray}
        &
        (3-1)(4-1) = 6
        \\
        \text{Residual}
        &
        \begin{bNiceArray}{rr}
            312 & -84 \\
            -84 & 400
        \end{bNiceArray}
        &
        3(4)(2-1) = 12
        \\
        \hline \\
        \text{Total (corrected)}
        &
        \begin{bNiceArray}{rr}
            876 & 124 \\
            124 & 688
        \end{bNiceArray}
        &
        3(4)(2) - 1 = 23
    \end{array}
    \]
    \item Given the results in Part b, test for interactions, and if the interactions do not
    exist, test for factor 1 and factor 2 main effects. Use the likelihood ratio test with
    $\alpha= .05$.
    \begin{align*}
        \bm{\Lambda}^{\star} = \frac{|\text{SSP}_{res}|}{|\text{SSP}_{int} + \text{SSP}_{res}|}
        & =
        \frac{117744}{145680}
        =
        0.8082
    \end{align*}
    \begin{align*}
        \nu_{1}
        & =
        |(g-1)(b-1) - p| + 1 = |2(3) - 2| + 1 = 5
    \end{align*}
    \begin{align*}
        \nu_{2}
        & =
        gb(n-1) - p + 1 = 3(4)(1) - 2 + 1 = 11
    \end{align*}
    \begin{align*}
        F
        & =
        \left(\frac{1 - \bm{\Lambda}^{\star}}{\bm{\Lambda^{\star}}}\right)\frac{\nu_{2}/2}{\nu_{1}/2}
        =
        \left(\frac{1 - 0.8082}{0.8082}\right)\frac{5.5}{2.5}
        =
        0.52
    \end{align*}
    \begin{align*}
        F_{\nu_{1}, \nu_{2}}(\alpha)
        =
        F_{5, 11}(0.05)
        =
        3.20
    \end{align*}
    We have that $F = 0.52 < F_{\text{crit}} = F_{5, 11}(0.05) = 3.20$, so we would fail to reject the null hypothesis that $\bm{\gamma}_{1,1}=\bm{\gamma}_{1,2}=\bm{\gamma}_{1,3}=\bm{\gamma}_{1,4}=\bm{\gamma}_{2,1}=\bm{\gamma}_{2,2}=\bm{\gamma}_{2,3}=\bm{\gamma}_{2,4}=\bm{\gamma}_{3,1}=\bm{\gamma}_{3,2}=\bm{\gamma}_{3,3}=\bm{\gamma}_{3,4}=\textbf{0}$. Okay, it doesn't look like we have interaction, so it's safe to test for main effects.
    \begin{align*}
        \bm{\Lambda}_{1}^{\star} = \frac{|\text{SSP}_{res}|}{|\text{SSP}_{fac1} + \text{SSP}_{res}|}
        & =
        \frac{117744}{481264}
        =
        0.2447
    \end{align*}
    \begin{align*}
        \nu_{1}
        & =
        |(g-1) - p| + 1 = |2 - 2| + 1 = 1
    \end{align*}
    \begin{align*}
        \nu_{2}
        & =
        gb(n-1) - p + 1 = 3(4)(1) - 2 + 1 = 11
    \end{align*}
    \begin{align*}
        F_{1}
        & =
        \left(\frac{1 - \bm{\Lambda}_{1}^{\star}}{\bm{\Lambda_{1}^{\star}}}\right)\frac{\nu_{2}/2}{\nu_{1}/2}
        =
        \left(\frac{1 - 0.2447}{0.2447}\right)\frac{5.5}{0.5}
        =
        33.96
    \end{align*}
    \begin{align*}
        F_{\nu_{1}, \nu_{2}}(\alpha)
        =
        F_{1, 11}(0.05)
        =
        4.84
    \end{align*}
    We have that $F_{1} = 33.96 > F_{\text{crit}} = F_{1, 11}(0.05) = 4.84$, so we would reject the null hypothesis that $\bm{\tau}_{1}=\bm{\tau}_{2}=\bm{\tau}_{3}=\textbf{0}$, and conclude there are effects for Factor 1.
    \begin{align*}
        \bm{\Lambda}_{2}^{\star} = \frac{|\text{SSP}_{res}|}{|\text{SSP}_{fac2} + \text{SSP}_{res}|}
        & =
        \frac{117744}{148128}
        =
        0.7949
    \end{align*}
    \begin{align*}
        \nu_{1}
        & =
        |(b-1) - p| + 1 = |3 - 2| + 1 = 2
    \end{align*}
    \begin{align*}
        \nu_{2}
        & =
        gb(n-1) - p + 1 = 3(4)(1) - 2 + 1 = 11
    \end{align*}
    \begin{align*}
        F_{1}
        & =
        \left(\frac{1 - \bm{\Lambda}_{2}^{\star}}{\bm{\Lambda_{2}^{\star}}}\right)\frac{\nu_{2}/2}{\nu_{1}/2}
        =
        \left(\frac{1 - 0.7949}{0.7949}\right)\frac{5.5}{1.0}
        =
        1.42
    \end{align*}
    \begin{align*}
        F_{\nu_{1}, \nu_{2}}(\alpha)
        =
        F_{2, 11}(0.05)
        =
        3.98
    \end{align*}
    We have that $F_{1} = 1.42 < F_{\text{crit}} = F_{2, 11}(0.05) = 3.98$, so we would fail to reject the null hypothesis that $\bm{\beta}_{1}=\bm{\beta}_{2}=\bm{\beta}_{3}=\bm{\beta}_{4}=\textbf{0}$, and conclude there are no effects for Factor 2.

    \item If main effects, but no interactions, exist, examine the nature of the main effects by
    constructing Bonferroni simultaneous 95\% confidence intervals for differences of
    the components of the factor effect parameters.

    Only Factor 1 was significant, so here, I created the Bonferroni CI's only for Factor 1. 
    The Factor 2 CI's all contained zero anyway. We can see that for variable 1, the diffrence between treatment 1 and treatment 3 is the only difference that does not contain zero. It's positive, so treatment 1 is larger than treatment 2 and the difference is somewhere between 2.96 and 19.04.

    \begin{align*}
        \nu = gb(n-1) & = 3(4)(2-1) = 12
    \end{align*}

    \begin{align*}
        t_{\nu} \left( \frac{\alpha}{pg(g-1)} \right)
        & =
        t_{12} \left( \frac{0.05}{2(3)(3-1)} \right)
        \\
        & =
        t_{12} \left( \frac{0.05}{12} \right)
        \\
        & =
        3.152681
    \end{align*}

    \begin{align*}
        t_{\nu} \left( \frac{\alpha}{pb(b-1)} \right)
        & =
        t_{12} \left( \frac{0.05}{2(4)(4-1)} \right)
        \\
        & =
        t_{12} \left( \frac{0.05}{24} \right)
        \\
        & =
        3.527352
    \end{align*}

    \[
        E_{11} = 312
        \hspace{0.20cm}
        \text{and}
        \hspace{0.20cm}
        E_{22} = 400
    \]

    \underline{Variable 1:}
    \begin{align*}
        \ell=1, m=2, i=1
        &
        \\
        \tau_{11} - \tau_{21} \pm t_{\nu} \left( \frac{\alpha}{pg(g-1)} \right) \sqrt{\frac{E_{11}}{\nu}\frac{2}{bn}}
        & = [-1.0378, 15.0378]
    \end{align*}
    \begin{align*}
        \ell=1, m=3, i=1
        &
        \\
        \tau_{11} - \tau_{31} \pm t_{\nu} \left( \frac{\alpha}{pg(g-1)} \right) \sqrt{\frac{E_{11}}{\nu}\frac{2}{bn}}
        & = [2.9622, 19.0378]
    \end{align*}
    \begin{align*}
        \ell=2, m=3, i=1
        &
        \\
        \tau_{21} - \tau_{31} \pm t_{\nu} \left( \frac{\alpha}{pg(g-1)} \right) \sqrt{\frac{E_{11}}{\nu}\frac{2}{bn}}
        & = [-4.0378, 12.0378]
    \end{align*}

    \underline{Variable 2:}
    \begin{align*}
        \ell=1, m=2, i=2
        &
        \\
        \tau_{11} - \tau_{21} \pm t_{\nu} \left( \frac{\alpha}{pg(g-1)} \right) \sqrt{\frac{E_{22}}{\nu}\frac{2}{bn}}
        & = [-11.1010, 7.1010]
    \end{align*}
    \begin{align*}
        \ell=1, m=3, i=2
        &
        \\
        \tau_{11} - \tau_{21} \pm t_{\nu} \left( \frac{\alpha}{pg(g-1)} \right) \sqrt{\frac{E_{22}}{\nu}\frac{2}{bn}}
        & = [-4.1010, 14.1010]
    \end{align*}
    \begin{align*}
        \ell=2, m=2, i=2
        &
        \\
        \tau_{11} - \tau_{21} \pm t_{\nu} \left( \frac{\alpha}{pg(g-1)} \right) \sqrt{\frac{E_{22}}{\nu}\frac{2}{bn}}
        & = [-2.1010, 16.1010]
    \end{align*}
\end{enumerate}