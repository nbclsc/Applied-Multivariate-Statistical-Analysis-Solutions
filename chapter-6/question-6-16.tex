Four measures of the response stiffness on each of 30 boards are listed in Table 4.3 (see Example 4.14).
The measures, on a given board, are repeated in the sense that they were
made one after another.
Assuming that the measures of stiffness arise from four treatments, test for the equality of treatments in a repeated measures design context.
Set  $\alpha = .05$.
Construct a 95\% (simultaneous) confidence interval for a contrast in the
mean levels representing a comparison of the dynamic measurements with the static measurements.

\[
    \bar{\textbf{x}}
    =
    \begin{bNiceArray}{r}
        -1906.1    \\ 
        -1749.5333 \\
        -1509.1333 \\
        -1724.9667
    \end{bNiceArray}
\]
\[
    \textbf{S}
    =
    \begin{bNiceArray}{rrrr}
        105616.300 &  94613.531 & 87289.7103 & 94230.7276  \\
        94613.5310 & 101510.119 & 76137.0989 & 81064.3632  \\
        87289.7103 &  76137.098 & 91917.0851 & 90352.3839  \\
        94230.7276 &  81064.363 & 90352.3839 & 104227.9644
    \end{bNiceArray}
\]
\[
    \textbf{C}
    =
    \begin{bNiceArray}{rrrr}
        1 & -1 &  0 &  0 \\
        0 &  1 & -1 &  0 \\
        0 &  0 &  1 & -1
    \end{bNiceArray}
\]
From 6.2, the test for equality of treatments in a repeated measure design (page 280), we have formula (6--16). Replacing $q$ with $p$, we can plug-in all the bits
\[
    T^{2}
    =
    n
    {\left(\textbf{C}\bar{\textbf{x}}\right)}^{\prime}
    {\left(\textbf{C}\textbf{S}\textbf{C}^{\prime}\right)}^{-1}
    {\left(\textbf{C}\bar{\textbf{x}}\right)}
    =
    254.7212
\]
\begin{align*}
    f_{\text{crit}}
    & =
    \frac{(n-1)(p-1)}{n-p+1}
    F_{p-1, n-p+1}
    (\alpha)
    \\
    & =
    \frac{(30-1)(4-1)}{30-4+1}
    F_{4-1, 30-4+1}
    (0.05)
    \\
    & =
    3.2222 \times 2.9604
    \\
    & =
    9.5389
\end{align*}
We have that $T^{2} = 254.7212 > f_{\text{crit}} = 9.5389$, so we would reject the null hypothesis that $\textbf{C}\bm{\mu} = \textbf{0}$, so there are differences among the means.

To compute the 95\% simultaneous confidence intervals, we use the formula (6-18) on page 281.

\[
    \textbf{c}
    =
    \begin{bNiceArray}{r}
         1 \\
         1 \\
        -1 \\
        -1 \\
    \end{bNiceArray}
\]

\begin{align*}
    \textbf{c}^{\prime}\bm{\mu}
    & :
    \textbf{c}^{\prime}\hat{\textbf{x}}
    \pm
    \sqrt{\frac{(n-1)(p-1)}{n-q+1} F_{p-1, n-p+1} (\alpha)}
    \sqrt{\frac{\textbf{c}^{\prime}\textbf{S}\textbf{c}}{n}}
    \\
    & \Rightarrow
    421.53 \pm \sqrt{3.2222 \times 2.9604}\sqrt{3191.9833}
    \\
    & \Rightarrow
    421.53 \pm \sqrt{9.5389}\sqrt{3191.9833}
    \\
    & \Rightarrow
    421.53 \pm 174.4937 
    \\
    & \Rightarrow
    (247.0397, 596.0270) 
\end{align*}