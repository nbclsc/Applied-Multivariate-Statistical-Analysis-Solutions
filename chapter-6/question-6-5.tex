A researcher considered three indices measuring the severity of heart attacks. The
values of these indices for $n = 40$ heart-attack patients arriving at a hospital emergency room produced the summary statistics
\[
    \bar{\textbf{x}}
    =
    \begin{bNiceArray}{c}
        46.1 \\
        57.3 \\
        50.4
    \end{bNiceArray}
    \hspace{0.20cm}
    \text{and}
    \hspace{0.20cm}
    \textbf{S}
    =
    \begin{bNiceArray}{ccc}
        101.3 & 63.0 & 71.0 \\
        63.0  & 80.2 & 55.6 \\
        71.0  & 55.6 & 97.4
    \end{bNiceArray}
\]

\begin{enumerate}[label= (\alph*)]
    \item All three indices are evaluated for each patient. Test for the equality of mean indices
    using (6--16) with $\alpha = .05$.

    \begin{align*}
        H_{0}: \textbf{C} \bm{\mu} &= \textbf{0} \\
        H_{1}: \textbf{C} \bm{\mu} &\ne \textbf{0}
    \end{align*}

    \[
        \textbf{C}
        =
        \begin{bNiceArray}{rrr}
            -1 &  1 & 0 \\
             0 & -1 & 1
        \end{bNiceArray}
    \]

    \begin{align*}
        \textbf{C}\bar{\textbf{x}}
        =
        \begin{bNiceArray}{rrr}
            -1 &  1 & 0 \\
             0 & -1 & 1
        \end{bNiceArray}
        \begin{bNiceArray}{c}
            46.1 \\
            57.3 \\
            50.4
        \end{bNiceArray}
        =
        \begin{bNiceArray}{r}
            11.2 \\
            -6.9
        \end{bNiceArray}
    \end{align*}

    \begin{align*}
        \textbf{C}\textbf{S}\textbf{C}^{\prime}
        &=
        \begin{bNiceArray}{rrr}
            -1 &  1 & 0 \\
             0 & -1 & 1
        \end{bNiceArray}
        \begin{bNiceArray}{ccc}
            101.3 & 63.0 & 71.0 \\
            63.0  & 80.2 & 55.6 \\
            71.0  & 55.6 & 97.4
        \end{bNiceArray}
        \begin{bNiceArray}{rr}
            -1 &  0 \\
             1 & -1 \\
             0 &  1
        \end{bNiceArray} \\
        &=
        \begin{bNiceArray}{rr}
            55.5  & -32.6 \\
            -32.6 &  66.4
        \end{bNiceArray}
    \end{align*}

    \begin{align*}
        T^{2}
        &=
        n{(\textbf{C}\bar{\textbf{x}})}^{\prime}
        {(\textbf{C}\textbf{S}\textbf{C}^{\prime})}^{-1}
        \textbf{C}\bar{\textbf{x}} \\
        &=
        40
        \begin{bNiceArray}{rr}
            11.2 & -6.9
        \end{bNiceArray}
        {\left(\begin{bNiceArray}{rr}
            55.5  & -32.6 \\
            -32.6 &  66.4
        \end{bNiceArray}\right)}^{-1}
        \begin{bNiceArray}{r}
            11.2 \\
            -6.9
        \end{bNiceArray} \\
        &=
        90.4946
    \end{align*}

    \begin{align*}
        F_{\text{crit}}
        &=
        \frac{ (n-1)(q-1)}{n-(q-1) } F_{ q-1, n-(q-1) }(\alpha) \\
        &=
        \frac{{ (40-1)(3-1) }}{{ 40-(3-1) }} F_{{ 3-1, 40-(3-1) }}(0.05) \\
        &=
        2.05 (3.24) \\
        &=
        6.6604
    \end{align*}

    We have that $T^{2} = 90.4946 > F_{\text{crit}} = F_{2, 38}(0.05) =
    6.6604$, so we would reject the null hypothesis that
    $\textbf{C} \bm{\mu} = \textbf{0}$ (equality of mean indices).
    
    \item Judge the differences in pairs of mean indices using 95\% simultaneous confidence
    intervals. [See (6--18).]

    $\textbf{c}_{1}^{\prime}\bm{\mu} = \mu_{2} - \mu_{1}$

    \begin{align*}
        \textbf{c}_{1}^{\prime}\bar{\textbf{x}}
        & \pm
        \sqrt{ \frac{ (n-1)(q-1)}{n-(q-1) } F_{ q-1, n-(q-1) }(\alpha) }
        \sqrt{ \frac{\textbf{c}_{1}^{\prime}\textbf{S}\textbf{c}_{1}}{n} } = \\
        11.2
        & \pm
        \sqrt{{ 6.6604 }}
        \sqrt{ \frac{55.5}{40} } = \\
        11.2
        & \pm
        3.04 \hspace{0.2cm}\text{or}\hspace{0.2cm}(8.160, 14.240)
    \end{align*}

    $\textbf{c}_{2}^{\prime}\bm{\mu} = \mu_{3} - \mu_{2}$

    \begin{align*}
        \textbf{c}_{2}^{\prime}\bar{\textbf{x}}
        & \pm
        \sqrt{ \frac{ (n-1)(q-1)}{n-(q-1) } F_{ q-1, n-(q-1) }(\alpha) }
        \sqrt{ \frac{\textbf{c}_{2}^{\prime}\textbf{S}\textbf{c}_{2}}{n} } = \\
        -6.90
        & \pm
        \sqrt{{ 6.6604 }}
        \sqrt{ \frac{66.4}{40} } = \\
        -6.90
        & \pm
        3.04 \hspace{0.2cm}\text{or}\hspace{0.2cm}(-10.225, -3.575)
    \end{align*}

    $\textbf{c}_{3}^{\prime}\bm{\mu} = \mu_{3} - \mu_{1}$

    I threw in this third contrast even though it's redundent from a spanning perspective, i.e., row1 + row2 = [-1,0,1].

    \begin{align*}
        \textbf{c}_{3}^{\prime}\bar{\textbf{x}}
        & \pm
        \sqrt{ \frac{ (n-1)(q-1)}{n-(q-1) } F_{ q-1, n-(q-1) }(\alpha) }
        \sqrt{ \frac{\textbf{c}_{3}^{\prime}\textbf{S}\textbf{c}_{3}}{n} } = \\
        4.30
        & \pm
        \sqrt{{ 6.6604 }}
        \sqrt{ \frac{56.7}{40} } = \\
        4.30
        & \pm
        3.04 \hspace{0.2cm}\text{or}\hspace{0.2cm}(1.227, 7.373)
    \end{align*}


    Since the 95\% CI of the first contrast, $\textbf{c}_{1}^{\prime}\bm{\mu}$, is positive, index 2 mean is significantly larger than index 1 mean.
    For the 95\% CI of the second contrast, $\textbf{c}_{2}^{\prime}\bm{\mu}$, the interval is negative, so the index 2 mean is also significantly larger than index 3 mean.
    For the 95\% CI of the third contrast, $\textbf{c}_{3}^{\prime}\bm{\mu}$, the interval is positive, so index 3 mean is significantly larger than index 1 mean.

\end{enumerate}