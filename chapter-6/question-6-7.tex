Using the summary statistics for the electricity-demand data given in Example 6.4, compute
$T^{2}$ and test the hypothesis $H_{0}: \bm{\mu}_{1} - \bm{\mu}_{2} = \textbf{0}$, assuming that $\bm{\Sigma}_{1} = \bm{\Sigma}_{2}$.
Set $\alpha = .05$.
Also, determine the linear combination of mean components most responsible for the
rejection of $H_{0}$.

\begin{align*}
    \textbf{S}_{\text{pooled}}
    & =
    \frac{(n_{1} - 1)\textbf{S}_{1} + (n_{2} - 1)\textbf{S}_{2}}{(n_{1} - 1) + (n_{2} - 1)} \\
    & =
    \begin{bNiceArray}{cc}
        10963.6857 & 21505.4224 \\
        21505.4224 & 63661.3122
    \end{bNiceArray}
\end{align*}

\begin{align*}
    T^{2}
    & =
    {[\bar{\textbf{x}}_{1} - \bar{\textbf{x}}_{2} - (\bm{\mu}_{1} - \bm{\mu}_{2})]}^{\prime}
    {\left[
        \left(
            \frac{1}{n_{1}}
            +
            \frac{1}{n_{2}}
        \right)
        \textbf{S}_{\text{pooled}}
    \right]}^{-1}
    [\bar{\textbf{x}}_{1} - \bar{\textbf{x}}_{2} - (\bm{\mu}_{1} - \bm{\mu}_{2})] \\
    & =
    \begin{bNiceArray}{cc}
        74.4 & 201.6
    \end{bNiceArray}
    {\left[
        \left(
            \frac{1}{45}
            +
            \frac{1}{55}
        \right)
        \begin{bNiceArray}{cc}
            10963.6857 & 21505.4224 \\
            21505.4224 & 63661.3122
        \end{bNiceArray}
    \right]}^{-1}
    \begin{bNiceArray}{c}
        74.4 \\
       201.6
    \end{bNiceArray} \\
    & =
    16.0662
\end{align*}

\begin{align*}
    c^{2}
    & =
    \frac{(n_{1} + n_{2} - 2)p}{(n_{1} + n_{2} - (p + 1))}
    F_{p, n_{1} + n_{2} - (p + 1)}(\alpha) \\
    & =
    \frac{(98)2}{(97)}
    F_{2, 97}(0.05) \\
    & =
    2.0206 \times 3.0902 \\
    & =
    6.2441
\end{align*}

We have that $T^{{2}} = 16.0662 > c^{2} = F_{2, 97}(0.05) = 6.2441$, so we would reject the null hypothesis that
$\bm{\mu}_{1} = \bm{\mu}_{2}$ (the mean vectors for the two groups, those with and without AC, are equal).

\begin{align*}
    \hat{\textbf{a}}
    & \propto
    \textbf{S}^{-1}_{\text{pooled}}
    (\bar{\textbf{x}}_{1} - \bar{\textbf{x}}_{2}) \\
    & =
    \begin{bNiceArray}{rr}
         0.00027035 & -0.00009133 \\
        -0.00009133 &  0.00004656
    \end{bNiceArray}
    \begin{bNiceArray}{c}
         74.4 \\
        201.6
    \end{bNiceArray} \\
    & =
    \begin{bNiceArray}{c}
        0.00170252 \\
        0.00259163
   \end{bNiceArray}
\end{align*}

Okay, so this is the direction in which the two groups (with and without AC) differ the most when accounting for the covariance structure of the data, as modeled by $\text{S}_{\text{pooled}}$.
This is a weighted linear combination of on-peak and off-peak consumption.
I'm thinking of this as a projection of the mean difference vector onto the pooled sample covariance matrix,
or applying the pooled sample covariance matrix as a transformation to the mean difference vector.
From the output, it looks like the off-peak consumption ($X_{2}$) has more weight in the largest difference, than on-peak consumption ($X_{1}$).