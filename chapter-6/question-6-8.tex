Observations on two responses are collected for three treatments. The observation
vectors $\begin{bNiceArray}{c}x_{1} \\ x_{2}\end{bNiceArray}$ are

\[
\begin{NiceArray}{cccccc}[cell-space-top-limit = 0.25cm,cell-space-bottom-limit = 0.25cm]
    \text{Treatment 1:}
    &
    \left[
        \begin{array}{c}
            6 \\
            7
        \end{array}
    \right],
    &
    \left[
        \begin{array}{c}
            5 \\
            9
        \end{array}
        \right],
    &
    \left[
        \begin{array}{c}
            8 \\
            6
        \end{array}
        \right],
    &
    \left[
        \begin{array}{c}
            4 \\
            9
        \end{array}
    \right],
    &
    \left[
        \begin{array}{c}
            7 \\
            9
        \end{array}
    \right]
    \\
    \text{Treatment 2:}
    &
    \left[
        \begin{array}{c}
            3 \\
            3
        \end{array}
    \right],
    &
    \left[
        \begin{array}{c}
            1 \\
            6
        \end{array}
    \right],
    &
    \left[
        \begin{array}{c}
            2 \\
            3
        \end{array}
    \right]\phantom{,}
    &
    &
    \\
    \text{Treatment 3:}
    &
    \left[
        \begin{array}{c}
            2 \\
            3
        \end{array}
    \right],
    &
    \left[
        \begin{array}{c}
            5 \\
            1
        \end{array}
    \right],
    &
    \left[
        \begin{array}{c}
            3 \\
            1
        \end{array}
    \right],
    &
    \left[
        \begin{array}{c}
            2 \\
            3
        \end{array}
    \right]\phantom{,}
    &
\end{NiceArray}
\]

\begin{enumerate}[label= (\alph*)]
    \item Break up the observations into mean, treatment, and residual components, as in
    (6--39). Construct the corresponding arrays for each variable. (See Example 6.9.)

    \[
        \text{Overall sample mean vector: }
        \bar{\textbf{x}}
        =
        \begin{bNiceArray}{c}
            4 \\
            5
        \end{bNiceArray}
    \]
    \[
        \text{Sample mean vectors for each group: }
        \bar{\textbf{x}}_{1}
        =
        \begin{bNiceArray}{c}
            6 \\
            8
        \end{bNiceArray}
        \text{,}
        \hspace{0.4cm}
        \bar{\textbf{x}}_{2}
        =
        \begin{bNiceArray}{c}
            2 \\
            4
        \end{bNiceArray}
        \text{,}
        \hspace{0.4cm}
        \bar{\textbf{x}}_{3}
        =
        \begin{bNiceArray}{c}
            3 \\
            2
        \end{bNiceArray}
    \]

    From (6--39)
    \begin{multline*}
        \begin{array}{c}
            \textbf{x}_{\ell j} \\
            \text{(observation)}
        \end{array}
        =
        \begin{array}{c}
            \bar{\textbf{x}} \\
            \left(
                \begin{array}{c}
                    \text{overall sample} \\
                    \text{mean } \hat{\bm{\mu}}
                \end{array}
            \right)
        \end{array}
        + \\
        \begin{array}{c}
            (\textbf{x}_{\ell} - \bar{\textbf{x}}) \\
            \left(
                \begin{array}{c}
                    \text{estimated} \\
                    \text{treatment} \\
                    \text{effect } \hat{\bm{\tau}}_{\ell}
                \end{array}
            \right)
        \end{array}
        +
        \begin{array}{c}
            (\textbf{x}_{\ell j} - \textbf{x}_{\ell}) \\
            \left(
                \begin{array}{c}
                    \text{residual } \\
                    \text{effect } \hat{\textbf{e}}_{\ell j}
                \end{array}
            \right)
        \end{array}
    \end{multline*}

    \begin{multline*}
        \begin{array}{c}
            \left[
                \begin{array}{ccccc}
                    6 & 5 & 8 & 4 & 7 \\
                    3 & 1 & 2 &   &   \\
                    2 & 5 & 3 & 2 & 
                \end{array}
            \right] \\
            \text{(observations)}
        \end{array}
        =
        \begin{array}{c}
            \left[
                \begin{array}{ccccc}
                    4 & 4 & 4 & 4 & 4 \\
                    4 & 4 & 4 &   &   \\
                    4 & 4 & 4 & 4 & 
                \end{array}
            \right] \\
            \text{(mean)}
        \end{array}
        +\\
        \begin{array}{c}
            \left[
                \begin{array}{rrrrr}
                    2 &  2 &  2 &  2 & 2 \\
                    -2 & -2 & -2 &    &   \\
                    -1 & -1 & -1 & -1 & 
                \end{array}
            \right] \\
            \text{(treatment effect)}
        \end{array}
        +
        \begin{array}{c}
            \left[
                \begin{array}{rrrrr}
                    0 & -1 & 2 & -2 & 1 \\
                    1 & -1 & 0 &    &   \\
                    -1 &  2 & 0 & -1 & 
                \end{array}
            \right] \\
            \text{(residual)}
        \end{array}
    \end{multline*}

    \begin{multline*}
        \begin{array}{c}
            \left[
                \begin{array}{ccccc}
                    7 & 9 & 6 & 9 & 9 \\
                    3 & 6 & 3 &   &   \\
                    3 & 1 & 1 & 3 & 
                \end{array}
            \right] \\
            \text{(observations)}
        \end{array}
        =
        \begin{array}{c}
            \left[
                \begin{array}{ccccc}
                    5 & 5 & 5 & 5 & 5 \\
                    5 & 5 & 5 &   &   \\
                    5 & 5 & 5 & 5 & 
                \end{array}
            \right] \\
            \text{(mean)}
        \end{array}
        + \\
        \begin{array}{c}
            \left[
                \begin{array}{ccccc}
                    3 &  3 &  3 &  3 & 3 \\
                    -1 & -1 & -1 &    &   \\
                    -3 & -3 & -3 & -3 & 
                \end{array}
            \right] \\
            \text{(treatment effect)}
        \end{array}
        +
        \begin{array}{c}
            \left[
                \begin{array}{ccccc}
                    -1 &  1 & -2 & 1 & 1 \\
                    -1 &  2 & -1 &   &   \\
                    1 & -1 & -1 & 1 & 
                \end{array}
            \right] \\
            \text{(residual)}
        \end{array}
    \end{multline*}

    \item Using the information in Part a,construct the one-way MANOVA table.
    
    Okay, what's displayed above is, $\textbf{X}_{v} = \textbf{M}_{v} + \textbf{T}_{v} + \textbf{E}_{v}$, where $v$ identifies which measurement we're looking at.
For the sum of squares and cross-products in the MANOVA table we need a matrix result, that comes from some block computations. The book doesn't explicitly show the notation for it, but here's what the computations look like:

    \[
        \textbf{B}
        =
        \left[
            \begin{array}{cc}
                \text{sum}(\textbf{T}_{1} \circ \textbf{T}_{1}) & \text{sum}(\textbf{T}_{1} \circ \textbf{T}_{2}) \\
                \text{sum}(\textbf{T}_{2} \circ \textbf{T}_{1}) & \text{sum}(\textbf{T}_{2} \circ \textbf{T}_{2})
            \end{array}
        \right]
        =
        \left[
            \begin{array}{cc}
                \text{tr}(\textbf{T}_{1} \textbf{T}_{1}^{\prime}) & \text{tr}(\textbf{T}_{1} \textbf{T}_{2}^{\prime}) \\
                \text{tr}(\textbf{T}_{2} \textbf{T}_{1}^{\prime}) & \text{tr}(\textbf{T}_{2} \textbf{T}_{2}^{\prime})
            \end{array}
        \right]
    \]
    \begin{align*}
        \text{tr}(\textbf{T}_{1} \textbf{T}_{1}^{\prime})
        & =
        \text{tr}
        \left(
            \begin{bNiceArray}{rrrrr}
                2 &  2 &  2 &  2 & 2 \\
                -2 & -2 & -2 &  0 & 0 \\
                -1 & -1 & -1 & -1 & 0
            \end{bNiceArray}
            \begin{bNiceArray}{rrr}
                2 & -2 & -1 \\
                2 & -2 & -1 \\
                2 & -2 & -1 \\
                2 &  0 & -1 \\
                2 &  0 &  0
            \end{bNiceArray}
    \right) \\
    & =
    \text{tr}
        \left(
            \begin{bNiceArray}{rrr}
                20 & -12 & 8 \\
                -12 &  12 & 6 \\
                8 &   6 & 4
            \end{bNiceArray}
        \right) \\
        & =
        36
    \end{align*}

    \begin{align*}
        \text{tr}(\textbf{T}_{1} \textbf{T}_{2}^{\prime})
        & =
        \text{tr}(\textbf{T}_{2} \textbf{T}_{1}^{\prime}) \\
        & =
        \text{tr}
        \left(
            \begin{bNiceArray}{rrrrr}
                2 &  2 &  2 &  2 & 2 \\
                -2 & -2 & -2 &  0 & 0 \\
                -1 & -1 & -1 & -1 & 0
            \end{bNiceArray}
            \begin{bNiceArray}{rrr}
                3 & -1 & -3 \\
                3 & -1 & -3 \\
                3 & -1 & -3 \\
                3 &  0 & -3 \\
                3 &  0 &  0
            \end{bNiceArray}
    \right) \\
    & =
    \text{tr}
        \left(
            \begin{bNiceArray}{rrr}
                30 & -6 & -24 \\
                -18 &  6 &  18 \\
                -12 &  3 &  12
            \end{bNiceArray}
        \right) \\
        & =
        48
    \end{align*}

    \begin{align*}
        \text{tr}(\textbf{T}_{2} \textbf{T}_{2}^{\prime})
        & =
        \text{tr}
        \left(
            \begin{bNiceArray}{rrrrr}
                3 &  3 &  3 &  3 & 3 \\
                -1 & -1 & -1 &  0 & 0 \\
                -3 & -3 & -3 & -3 & 0
            \end{bNiceArray}
            \begin{bNiceArray}{rrr}
                3 & -1 & -3 \\
                3 & -1 & -3 \\
                3 & -1 & -3 \\
                3 &  0 & -3 \\
                3 &  0 &  0
            \end{bNiceArray}
    \right) \\
    & =
    \text{tr}
        \left(
            \begin{bNiceArray}{rrr}
                45 & -9 & -27 \\
                -9 &  3 &   9 \\
                -27 &  9 &  36
            \end{bNiceArray}
        \right) \\
        & =
        84
    \end{align*}

    \[
        \text{Treatment}
        =
        \textbf{B}
        =
        \begin{bNiceArray}{cc}
            \text{tr}(\textbf{T}_{1} \textbf{T}_{1}^{\prime}) & \text{tr}(\textbf{T}_{1} \textbf{T}_{2}^{\prime}) \\
            \text{tr}(\textbf{T}_{2} \textbf{T}_{1}^{\prime}) & \text{tr}(\textbf{T}_{2} \textbf{T}_{2}^{\prime})
        \end{bNiceArray}
        =
        \begin{bNiceArray}{cc}
            36 & 48 \\
            48 & 84
        \end{bNiceArray}
    \]

    \[
        \textbf{W}
        =
        \left[
            \begin{array}{cc}
                \text{sum}(\textbf{E}_{1} \circ \textbf{E}_{1}) & \text{sum}(\textbf{E}_{1} \circ \textbf{T}_{2}) \\
                \text{sum}(\textbf{E}_{2} \circ \textbf{E}_{1}) & \text{sum}(\textbf{E}_{2} \circ \textbf{T}_{2})
            \end{array}
        \right]
        =
        \left[
            \begin{array}{cc}
                \text{tr}(\textbf{E}_{1} \textbf{E}_{1}^{\prime}) & \text{tr}(\textbf{E}_{1} \textbf{E}_{2}^{\prime}) \\
                \text{tr}(\textbf{E}_{2} \textbf{E}_{1}^{\prime}) & \text{tr}(\textbf{E}_{2} \textbf{E}_{2}^{\prime})
            \end{array}
        \right]
    \]

    \begin{align*}
        \text{tr}(\textbf{E}_{1} \textbf{E}_{1}^{\prime})
        & =
        \text{tr}
        \left(
            \begin{bNiceArray}{rrrrr}
                0 & -1 & 2 & -2 & 1 \\
                1 & -1 & 0 &  0 & 0 \\
                -1 &  2 & 0 & -1 & 0
            \end{bNiceArray}
            \begin{bNiceArray}{rrr}
                0 &  1 & -1 \\
                -1 & -1 &  2 \\
                2 &  0 &  0 \\
                -2 &  0 & -1 \\
                1 &  0 &  0
            \end{bNiceArray}
    \right) \\
    & =
    \text{tr}
        \left(
            \begin{bNiceArray}{rrr}
                10 &  1 &  0 \\
                1 &  2 & -3 \\
                0 & -3 &  6
            \end{bNiceArray}
        \right) \\
        & =
        18
    \end{align*}

    \begin{align*}
        \text{tr}(\textbf{E}_{1} \textbf{E}_{2}^{\prime})
        & =
        \text{tr}(\textbf{E}_{2} \textbf{E}_{1}^{\prime}) \\
        & =
        \text{tr}
        \left(
            \begin{bNiceArray}{rrrrr}
                0 & -1 & 2 & -2 & 1 \\
                1 & -1 & 0 &  0 & 0 \\
                -1 &  2 & 0 & -1 & 0
            \end{bNiceArray}
            \begin{bNiceArray}{rrr}
                -1 & -1 &  1 \\
                1 &  2 & -1 \\
                -2 & -1 & -1 \\
                1 &  0 &  1 \\
                1 &  0 &  0
            \end{bNiceArray}
    \right) \\
    & =
    \text{tr}
        \left(
            \begin{bNiceArray}{rrr}
                -6 & -4 & -3 \\
                -2 & -3 &  2 \\
                2 &  5 & -4
            \end{bNiceArray}
        \right) \\
        & =
        -13
    \end{align*}

    \begin{align*}
        \text{tr}(\textbf{E}_{2} \textbf{E}_{2}^{\prime})
        & =
        \text{tr}
        \left(
            \begin{bNiceArray}{rrrrr}
                -1 &  1 & -2 & 1 & 1 \\
                -1 &  2 & -1 & 0 & 0 \\
                1 & -1 & -1 & 1 & 0
            \end{bNiceArray}
            \begin{bNiceArray}{rrr}
                -1 & -1 &  1 \\
                1 &  2 & -1 \\
                -2 & -1 & -1 \\
                1 &  0 &  1 \\
                1 &  0 &  0
            \end{bNiceArray}
    \right) \\
    & =
    \text{tr}
        \left(
            \begin{bNiceArray}{rrr}
                8 &  5 &  1 \\
                5 &  6 & -2 \\
                1 & -2 &  4
            \end{bNiceArray}
        \right) \\
        & =
        18
    \end{align*}

    \[
        \text{Residual}
        =
        \textbf{W}
        =
        \begin{bNiceArray}{cc}
            \text{tr}(\textbf{E}_{1} \textbf{E}_{1}^{\prime}) & \text{tr}(\textbf{E}_{1} \textbf{E}_{2}^{\prime}) \\
            \text{tr}(\textbf{E}_{2} \textbf{E}_{1}^{\prime}) & \text{tr}(\textbf{E}_{2} \textbf{E}_{2}^{\prime})
        \end{bNiceArray}
        =
        \begin{bNiceArray}{rr}
            18 & -13 \\
            -13 &  18
        \end{bNiceArray}
    \]
    \[
        \text{Total}
        =
        \textbf{B} + \textbf{W}
        =
        \begin{bNiceArray}{cc}
            36 & 48 \\
            48 & 84
        \end{bNiceArray}
        +
        \begin{bNiceArray}{rr}
             18 & -13 \\
            -13 &  18
    \end{bNiceArray}
    =
    \begin{bNiceArray}{cc}
        54 &  35 \\
        35 & 102
        \end{bNiceArray}
    \]

    \[
        \begin{array}{lll}
            \text{Source} & \text{Matrix of sum of squares} &  \\
            \text{of variation} & \text{and cross products} & \text{Degrees of freedom} \\
            \hline \\
            \text{Treatment} & 
            \begin{bNiceArray}{cc}
                36 & 48 \\
                48 & 84
            \end{bNiceArray} & 
            3 - 1 = 2 \\ \\
        \text{Residual} & 
        \begin{bNiceArray}{rr}
             18 & -13 \\
            -13 &  18
    \end{bNiceArray} & 
        5 + 4 + 3 - 3 = 9 \\ \\
        \hline \\
        \text{Total (corrected)} & 
        \begin{bNiceArray}{cc}
            54 &  35 \\
            35 & 102
        \end{bNiceArray} & 
        11
        \end{array}
    \]
    
    \item Evaluate Wilks' lambda, $\bm{\Lambda}^{\star}$, and use Table 6.3 to test for treatment effects.
    Set $\alpha = .01$. Repeat the test using the chi-square approximation with Bartlett's correction.
    [See (6--43).] Compare the conclusions.
    
    \par

    Setup the test:
    
        \begin{NiceTabular}{l}
            $H_{0}: \bm{\tau}_{1} = \bm{\tau}_{2} = \bm{\tau}_{3} = \textbf{0}$ \\
            \text{versus} \\
            $H_{1}: \text{At leat one } \bm{\tau}_{i} = \bm{\tau}_{j} \ne \textbf{0}$, for $i \ne j$ and $i,j = 1,2,3$
        \end{NiceTabular}
    

    \[
        \bm{\Lambda}^{\star}
        =
        \frac{\left|\textbf{W}\right|}{\left|\textbf{B} + \textbf{W}\right|}
        =
        \frac{18(18) - {(-13)}^{2}}{54(102) - {(35)}^{2}}
        =
        \frac{155}{8156}
        =
        0.0362
    \]
    \begin{align*}
        F^{\star}
        & =
        \left(
            \frac{\sum{n_{\ell} - g - 1}}{g - 1}
        \right)
        \left(
            \frac{1 - \sqrt{\bm{\Lambda}^{\star}}}{\sqrt{\bm{\Lambda}^{\star}}}
        \right) \\
        & =
        \left(
            \frac{12 - 3 - 1}{3 - 1}
        \right)
        \left(
            \frac{1 - \sqrt{0.0362}}{\sqrt{0.0362}}
        \right) \\
        & =
        17.0266
    \end{align*}
    Since $p = 2$ and $g = 3$ using Table 6.3,
    \[
        \left(
            \frac{\sum{n_{\ell} - g - 1}}{g - 1}
        \right)
        \left(
            \frac{1 - \sqrt{\bm{\Lambda}^{\star}}}{\sqrt{\bm{\Lambda}^{\star}}}
        \right)
        \sim
        F_{2(g-1), 2(\sum{n_{\ell} - g - 1})}(0.01)
        =
        4.7726
    \]
    We have that $F^{\star} = 17.027 > F_{\text{crit}} = F_{4,16}(0.01) = 4.773$, so we would reject the null hypothesis that $\bm{\tau}_{1} = \bm{\tau}_{2} = \bm{\tau}_{3} = \textbf{0}$.

    The setup for Barlett's test is the same as that above. From (6--43), Bartlett's adjustment test statistic is
    \begin{align*}
        X^{\star}
        & =
        -(n - 1 - (p + g)/2)
        \ln
        \left(
            \frac{|\textbf{W}|}{|\textbf{B} + \textbf{W}|}
        \right) \\
        & =
        -(12 - 1 - (2 + 3)/2)
        \ln
        \left(
            0.0362
        \right) \\
        & = 
        28.2114
    \end{align*}
    The critical value to compare the Bartlett test statistic against is
    \[
        X_{\text{crit}}
        =
        \chi_{p(g-1)}^{2}
        =
        \chi_{2(3-1)}^{2}
        =
        13.2767
    \]
    For Bartlett's test we have, $X^{\star} = 28.2114 > X_{\text{crit}} = \chi_{4}^{2}(0.01) = 13.2767$, so we would reject the null hypothesis that $\bm{\tau}_{1} = \bm{\tau}_{2} = \bm{\tau}_{3} = \textbf{0}$. This is the same conclusion as the test above.
\end{enumerate}